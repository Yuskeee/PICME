%% The openany option is here just to remove the blank pages before a new chapter
\documentclass[11pt,openany]{book}

\title{Notas sobre Teoria dos Grupos para Programa de Iniciação Científica e Mestrado - PICME}
\author{Rodrigo Yuske Yamauchi}

\usepackage{pagenote}
\usepackage{enumitem}
% \usepackage{enumerate}
\usepackage{amsmath}
\usepackage{amssymb}
\usepackage{amsthm}
\usepackage{hyperref}
\usepackage[matrix,arrow]{xy}
\usepackage[brazil]{babel}
\usepackage[T1]{fontenc}
\usepackage{thmtools}
\usepackage{graphicx}
\usepackage{setspace}
\usepackage{geometry}
\usepackage{float}
\usepackage[utf8]{inputenc}
\usepackage{framed}
\usepackage[dvipsnames]{xcolor}
\usepackage{tcolorbox}
\usepackage{enumitem}

% use \hyphenation for wrong spaced words

\newcommand{\separate}{\vspace{1.5em}}

% isomporphic maps to
\newcommand\isomto{\stackrel{\sim}{\smash{\longrightarrow}\rule{0pt}{0.4ex}}}

%subgroup generated by subset
\newcommand{\gen}[1]{\ensuremath{\langle #1\rangle}}

\renewcommand{\proofname}{Prova}
\renewcommand{\notesname}{Observações}
\renewcommand{\chaptername}{Capítulo}
% \newcommand{\novo}[1]{\textcolor{red}{#1}}
% \newcommand{\ignorar}[1]{\textcolor{blue}{#1}}
\newenvironment{novo}{
    \color{red}
}{}
\newenvironment{ignorar}{
    \color{blue}
}{}

% \newtheorem{theorem}{Teorema}
% \newtheorem{lemma}{Lema}
% \newtheorem{definition}{Definição}
% \newtheorem{proposition}{Proposição}
% \newtheorem{corollary}{Corolário}

\colorlet{LightGray}{White!90!Periwinkle}
\colorlet{LightOrange}{Orange!15}
\colorlet{LightGreen}{Green!15}
\colorlet{LightYellow}{Yellow!15}
\colorlet{LightBlue}{Blue!15}

\declaretheoremstyle[name=Teorema,]{thmsty}
\declaretheorem[style=thmsty,numberwithin=section]{theorem}
\tcolorboxenvironment{theorem}{colback=LightGray}

\declaretheoremstyle[name=Proposição,]{prosty}
\declaretheorem[style=prosty,numberlike=theorem]{proposition}
\tcolorboxenvironment{proposition}{colback=LightOrange}

\declaretheoremstyle[name=Definição,]{defsty}
\declaretheorem[style=defsty,numberlike=theorem]{definition}
\tcolorboxenvironment{definition}{colback=LightGreen}

\declaretheoremstyle[name=Lema,]{lemsty}
\declaretheorem[style=lemsty,numberlike=theorem]{lemma}
\tcolorboxenvironment{lemma}{colback=LightYellow}

\declaretheoremstyle[name=Corolário,]{corsty}
\declaretheorem[style=corsty,numberlike=theorem]{corollary}
\tcolorboxenvironment{corollary}{colback=LightBlue}

%% End notes to be printed as sections at the
%% end of each chapter.
\renewcommand*{\notedivision}{\section*{\notesname}}
\renewcommand*{\pagenotesubhead}[1]{}


%%%%%%%%%%%%% For customising the endnote markers. Comment these out if you don't want them.
% To prefix each note number with the chapter number
\renewcommand{\thepagenote}{\thechapter-\arabic{pagenote}}

% To have a slightly different formatting for the endnote numbers in the text -- smaller text, sans-serif, square brackets
\renewcommand\notenumintext[1]{\space{\footnotesize\sffamily[FN-#1]}}

% To have a slightly different formatting for the endnote numbers in the notes section. Just the square brackets and sans-serif; normal size.
\renewcommand\notenuminnotes[1]{{\sffamily[FN-#1] }}

\renewcommand{\contentsname}{Sumário}

%%%%%%%%%%%%% End customisation


%% THIS LINE IS MANDATORY
\makepagenote

\begin{document}

\maketitle

\tableofcontents

\chapter{Introdução a grupos}

\begin{definition}
\label{def:grupo}
    Seja um conjunto $A$ e uma operação binária $A \cdot A \rightarrow A$, diz-se que $(A, \cdot)$ é um grupo quando são satisfeitas as condições necessárias a seguir:
    
    \begin{enumerate}[label=\Roman*)] %[I)]
        \item $\forall \ \ a,b,c \in A$, \ \ $a\cdot(b\cdot c) = (a\cdot b)\cdot c$ \ \ (associatividade);
        \item $\exists \ e \in A$ tal que $a\cdot e = a$ \ \ (elemento neutro);
        \item $\forall \ \ a \in A$, $\exists \ b \in A$, tal que $a\cdot b = e$ \ \ (elemento inverso).
    \end{enumerate}
\end{definition}

A partir deste momento, uma operação $a \cdot b$, tais que $a, b \in A$ e $(A,\cdot)$ é um grupo, também será denotada simplesmente por $ab$ e o grupo poderá ser denotado pelo seu conjunto, e.g., $A$ é um grupo.

    \separate

    \textbf{(Generalização da Associatividade)}
    Seja $A$ um grupo, mostraremos que para $a_0 \dots a_n \in A$, 
    \[(a_0 \dots a_s) (a_{s+1} \dots a_n) = (a_0 \dots a_r) (a_{r+1} \dots a_n),\]
    tal que $r,s \in \mathbb{N}$ e $0 < r < s < n$.

    Para $n = 2$, é evidente que o que queremos mostrar é verdadeiro, uma vez que \[(a_0a_1)a_2 = a_0(a_1a_2)\] é a própria condição de $A$ ser um grupo.

    Para $n > 2$ e por indução em $n$, suponhamos que $\forall n'$, tal que $n' < n$, seja verdade que 
    \[(a_0 \dots a_{s'}) (a_{s'+1} \dots a_{n'}) = (a_0 \dots a_{r'}) (a_{r'+1} \dots a_{n'}),\]
    onde $0 < r' < s' < n'$.

    Como $r < s < n$, temos pela hipótese da indução que, sendo $r' = r$, $s' = s-1$ e $n' = s$,
    \begin{align*}
        (a_0 \dots a_s) (a_{s+1} \dots a_n) &= \left((a_0 \dots a_{s-1})(a_s)\right) (a_{s+1} \dots a_n) \\
        &= \left((a_0 \dots a_r)(a_{r+1} \dots a_s)\right)(a_{s+1} \dots a_n)\\
        &= (a_0 \dots a_r)\left((a_{r+1} \dots a_s)(a_{s+1} \dots a_n)\right)\\
        &= (a_0 \dots a_r)(a_{r+1} \dots a_n),
    \end{align*}
    como queríamos provar. $\square$

    \vspace{1.5em}

    Enunciaremos o seguinte lema que nos será útil posteriormente:\\
    \begin{lemma}
    \label{lemma:1}
        Sejam $a,b,c \in A$, se \[b\cdot a=c\cdot a \ \ \Rightarrow \ \ b=c.\]
    \end{lemma}
    \begin{proof}
        Seja $a'$ o elemento inverso de $a$,
        \begin{align*}
            b\cdot a\cdot a' &= c\cdot a\cdot a'\\
            b\cdot e &= c\cdot e\\
            \therefore b&= c.
        \end{align*}
    \end{proof}

    \begin{proposition}
    \label{prop:comutatividade_unicidade}
        Sendo $(A, \cdot)$ um grupo, mostraremos agora a comutatividade e unicidade de $e \in A$, tal que $a\cdot e = a$, e de $a' \in A$, tal que $a\cdot a' = e$.\\
    
        Para a comutatividade do elemento inverso, temos que
        \begin{align*}
            aa' &= e \\
            &= a'(a')' \\
            &= (a'e)(a')' \\
            &= a'(e(a')') \\
            &= a'((aa')(a')') \\
            &= a'(a(a'(a')')) \\
            &= a'(ae) = a'a,
        \end{align*}
        como queríamos mostrar. $\square$
    
        Quanto a comutatividade do elemento neutro,
        \begin{align*}
            ae = a(a'a) = (aa')a = ea,
        \end{align*}
        como queríamos. $\square$
    
        Provaremos a unicidade do elemento neutro por contradição, seja $e' \not= e$ um elemento neutro do grupo, tem-se então que
        \begin{align*}
            e'a &= a \\
            &= ea,
        \end{align*}
        então, pelo Lema \ref{lemma:1}, $e'=e$ e entramos em contradição, como queríamos mostrar. $\square$
    
        A fim de provar a unicidade do elemento inverso, consideremos $b, b' \in A$, tais que ambos sejam elementos inversos de $a$. Assim,
        \begin{align*}
            b\cdot a &= e = b'\cdot a\\
            \therefore b &= b',
        \end{align*}
        pelo \hyperref[lemma:1]{lema acima} novamente, como queríamos mostrar. $\square$
    \end{proposition}

    A partir de agora, denotaremos por $a^{-1}$ o único elemento inverso de $a \in A$.

    Note que agora é possível \textbf{redefinir} o conceito de grupo já com a unicidade e comutatividade dos elementos neutro e inverso e com a generalidade da associatividade, visto que estes todos são consequências diretas da definição mais abstrata.

    \separate

    Alguns \textbf{exemplos} notáveis de grupos são descritos a seguir.

    \begin{enumerate}[label=Exemplo \arabic*),align=left]
        \item O conjunto dos inteiros com a operação usual de soma é um grupo infinito, i.e., com um número infinito de elementos. Tal conjunto é denotado por $(\mathbb{Z}, +)$.
        
        \item O conjunto das classes de equivalência módulo $n$, i.e., $\{\overline{0}, \dots, \overline{n-1}\}$ e a soma dessas classes denotada por $\bigoplus\limits_{n}$ formam um grupo $(\mathbb{Z}/n\mathbb{Z}, \bigoplus\limits_{n})$ finito de $n$ elementos. Esse exemplo de grupo será melhor definido e mais explorado posteriormente
        
        \item O conjunto das permutações\footnote{Relação de bijeção entre um dado conjunto $A$ e $\{0,\dots, |A-1|\}$.} de $n$ elementos com a operação de composição de funções $\circ$ é um grupo e é denotado por $S_n$.
        \par Assim, se $n=3$,
        \[S_3 = \left\{\binom{123}{123}, \binom{123}{213}, \binom{123}{321}, \binom{123}{132}, \binom{123}{231}, \binom{123}{312}\right\},\]
        onde a notação $\binom{123}{abc}$ representa a função tal que $f(1)=a$, $f(2) = b$ e $f(3) = c$.
    \end{enumerate}

    Por fim, definimos ainda o que é um grupo abeliano, um exemplo importante de grupo a ser estudado mais a frente.
    \begin{definition}
    \label{def:abeliano}
        Um grupo $A$ \textit{abeliano} ou \textit{comutativo} é um grupo em que a seguinte propriedade é satisfeita:
        \[ab = ba, \ \forall a,b \in A.\]
    \end{definition}

\chapter{Subgrupos}

    \begin{definition}
    \label{def:subgrupo}
        Seja $A$ um grupo e $H \subseteq A$ não vazio, então se $H$ com a mesma operação de $A$, tal que $H\cdot H \rightarrow H$, também é um grupo, o chamaremos de subgrupo de $A$ e denotaremos por $H \leq A$. Ou seja, para que $H$ seja um subgrupo de $A$ as seguintes condições devem ser satisfeitas:
        \begin{enumerate}[label=\Roman*)]
            \item $\forall \ \ a,b,c \in H$, \ \ $a\cdot(b\cdot c) = (a\cdot b)\cdot c$ \ \ (associatividade);
            \item $\exists \ e \in H$ tal que $a\cdot e = a$ \ \ (elemento neutro);
            \item $\forall \ \ a \in H$, $\exists \ b \in A$, tal que $a\cdot b = e$ \ \ (elemento inverso);
            \item $\forall a,b \in H$, $a\cdot b \in H$ \ \ (operação binária fechada).
        \end{enumerate}
    \end{definition}
    

    \begin{proposition}
    \label{prop:subgrupo}
        Seja $H \subseteq A$ não vazio. Então, $H \leq A$ se, e somente se, as seguintes condições são satisfeitas:
        \begin{enumerate}
            \item $\forall a,b \in H$, $a\cdot b \in H$;
            \item $\forall a \in H$, $a^{-1} \in H$.
        \end{enumerate}
    \end{proposition}
    \begin{proof}
        Sendo $H \leq A$, então a primeira condição é imediatamente satisfeita. E, como $a \in A$ tem um único elemento inverso $a^{-1} \in A$, se $a \in H$, então $a^{-1} \in H$ pela condição da existência de elemento inverso para que $H$ seja subgrupo. Reciprocamente, se $H$ satisfaz a primeira condição da proposição, claramente é satisfeita a condição de operação binária fechada de subgrupo. Ademais, como vale a associatividade para elementos de $A$ e $H \subseteq A$, então consequentemente vale a associatividade para elementos de $H$. Ora, e se existe elemento inverso para todo elemento de $h \in H$, $hh^{-1} = e$, tal que $e \in A$, e pela condição de operação binária fechada, $e \in H$. Assim, é garantido a existência de elemento neutro e inverso $\forall h$.
    \end{proof}

    \separate

    Alguns exemplos de subgrupos são descritos a seguir:
    \begin{enumerate}[label=Exemplo \arabic*),align=left]
        \item O subconjunto $\{e\}$ forma um subgrupo para todo grupo, onde $e$ é o elemento neutro do grupo.

        \item O subconjunto $H \subseteq A$, tal que $A \subseteq H$, i.e., o próprio conjunto $A$ é subgrupo de $A$.
    \end{enumerate}

    %Definimos agora o conceito de subgrupo gerado por um subconjunto.

    \begin{definition}
    \label{def:subgrupo_gerado}
        Seja $S \subseteq A$ um subconjunto não vazio, onde $A$ é um grupo. Definimos
        \[\gen{S} = \{s_0s_1s_2 \dots s_n \ | \ n \in \mathbb{N}, s_i \in S \text{ ou } s_i^{-1} \in S\}.\]

        Ademais, se $a \in A$, notaremos $\gen{\{a\}}$ diretamente como $\gen{a}$.
    \end{definition}

    \begin{proposition}
    \label{prop:subgrupo_gerado}
        Sejam $S \subseteq A$ um subconjunto não vazio e $A$ um grupo, então $\gen{S}$ é um subgrupo de $A$.
    \end{proposition}
    
    \begin{proof}
        Basta provar a proposição \ref{prop:subgrupo}.
        Seja $x,y \in \gen{S}$,
        \[x = a_0a_1\dots a_n\text{, com $a_i \in S$ ou $a_i^{-1} \in S$.}\]
        \[y = b_0b_1\dots b_m\text{, com $b_i \in S$ ou $b_i^{-1} \in S$.}\]
        Ora, $xy = a_0a_1\dots a_nb_0b_1\dots b_m$, tal que todos os fatores são elementos de $S$ ou são o inverso de um elemento de $S$. Ademais, $x^{-1} = a_0^{-1}a_1^{-1}\dots a_n^{-1}$, tal que todos os fatores são elementos de $S$ ou inverso de um elemento de $S$. Assim, $xy, x^{-1} \in \gen{S}$, como queríamos.
    \end{proof}

    Dessa forma, a partir de agora chamaremos $\gen{S}$ por \textbf{subgrupo gerado pelo subconjunto $S$}, onde $S$ é o \textbf{conjunto gerador}.
    

    \begin{definition}
    \label{def:ciclico}
        Um grupo é dito \textbf{cíclico} quando ele pode ser gerado por um elemento, i.e., $\exists a \in A$ tal que $A = \gen{a}$. Note que $\gen{a} = \{a^n \ | \ n \in \mathbb{Z}\}$.
    \end{definition}


    \begin{definition}
    \label{def:ordem}
        Chamaremos de \textbf{ordem} de um grupo $A$, o número de elementos de $A$, e será denotada por $|A|$. Além disso, se um grupo é gerado por um elemento $a$, a ordem de $a$ será a ordem do subgrupo gerado por $a$, i.e., $|a| = |\gen{a}|$.
    \end{definition}


    \begin{theorem}
    \label{theor:ordem_finita_identidade}
        Sejam $A$ um grupo e $a \in A$, tal que a ordem de $a$, $|a| = m$, é finita. Então $m$ é o menor inteiro positivo tal que $a^m = e$, onde $e$ é o elemento neutro de $A$.
    \end{theorem}
    
    \begin{proof}
        Primeiro mostraremos que, sendo $|a|$ finita, existe um inteiro positivo $k$ tal que $a^k = e$. Temos a seguinte generalização para \gen{a}:
        \[\gen{a} = \{a^n \ | \ n \in \mathbb{Z}\},\]
        tal que $|a| = m$.
        Assim, devem existir, sem perda de generalidade, $p, q \in \mathbb{Z}$, tal que $p > q$ e $a^p = a^q$. Ora, 
        \begin{align*}
            a^p \cdot a^{-q} &= a^q \cdot a^{-q}\\
            a^{p-q} &= e,
        \end{align*}
        portanto, como $p-q > 0$, $\exists k > 0$, tal que $a^k = e$.

        Agora, considere a sequência de potências de $a$: 
        \[e, a^1, a^2, \dots, a^{k'-1},\]
        onde $k'$ é o menor inteiro $k$, tal que $a^k = e$. Mostraremos que todos os elementos dessa sequência são distintos. Para $k'= 1$, há apenas um elemento, e é imediata a validade da afirmação. Para $k' > 1$, suponhamos, sem perda de generalidade, $p, q \in \mathbb{Z}_{\geq 0}$, tal que $q < p < k'$. Ora,
        \begin{align*}
            a^p \cdot a^{-q} &= a^q \cdot a^{-q}\\
            a^{p-q} &= e,
        \end{align*}
        porém, como $0 < p - q < k'$, entramos em contradição, já que $k'$ é o menor inteiro tal que essa relação é verdadeira. Portanto, a sequência de potências de $a$: \[e, a^1, a^2, \dots, a^{k'-1}\] possui todos elementos distintos.

        Por fim, basta mostrar que $k' = m$. Para isso, consideremos $n \in \mathbb{Z}$. Pelo algoritmo de Euclides, pode-se escrever $n = qk'+r$, tal que $0 \leq r < k'$. Assim,
        \[a^n = a^{qk'+r} = a^{qk'}\cdot a^r = e\cdot a^r = a^r.\]
        Isso significa que para qualquer $n$, $a^n$ encontra-se na sequência de elementos distintos enunciada acima. Portanto,
        \[\gen{a} = \{a^n \ | \ n \in \mathbb{Z}\} = \{e, a^1, a^2, \dots, a^{k'-1}\},\]
        e ainda, como $|a| = m$, $m = k'$, como queríamos mostrar.
    \end{proof}

    \chapter{Classes Laterais}

    \begin{definition}
    \label{def:classe_lateral}
        Sejam $S \leq A$, $A$ um grupo e $a \in A$, define-se como \textbf{classe lateral à esquerda de $S$ em $A$} o subconjunto de $A$
        \[aS = \{as \ | \ s \in S\}.\]
        (uma \textbf{classe lateral à direita é definida por $Sa = \{sa \ | \ s \in S\}$}).

        Analogamente, pode-se definir a seguinte relação
        \[y \sim_E a \ \Leftrightarrow \exists \ s \in S \ \ \text{tal que y = as}.\]
        Assim, a \textbf{classe lateral à esquerda de $S$} pode também ser escrita como
        \[aS = \{y \in A \ | \ y \sim_E a\}\]
    \end{definition}


    \begin{proposition}
        A relação \[y \sim_E a \ \Leftrightarrow \exists \ s \in S \ \ \text{tal que y = as}\] é uma relação de equivalência\footnote{Uma relação de equivalência é uma relação que seja reflexiva, simétrica e transitiva}.
    \end{proposition}
    
    \begin{proof}
        Primeiramente, provaremos que a relação é reflexiva, i.e., $a \sim_E a$. Ora, $\forall a \in A$,
        \[ae \in A,\]
        onde $e$ é o elemento identidade do subgrupo $S$.
    
        Mostraremos agora que a relação é simétrica, i.e., se $y \sim_E a$, então $a \sim_E y$. Ora, se $\exists s \in S$ tal que $y = as$, então
        \begin{align*}
            y &= as\\
            ys^{-1} &= ass^{-1}\\
            ys^{-1} &= a,
        \end{align*}
        onde $s^{-1} \in S$, uma vez que $S$ é um subgrupo.
    
        Finalmente, mostraremos que a relação é transitiva, i.e., se $y \sim_E a$ e $a \sim_E b$, onde $a,b \in A$, então $y \sim_E b$. Ora, se $\exists s \in S$ tal que $y = as$ e $\exists t \in S$ tal que $a = bt$, então
        \begin{align*}
            y &= as\\
            &= bts\\
            &= bu,
        \end{align*}
        onde $u = ts \in S$, uma vez que $S$ é um subgrupo.
        Assim, a relação apresentada é uma relação de equivalência, como queríamos mostrar.
    \end{proof}


    \begin{lemma}
    \label{lemma:equivalencia_particao}
        Se $\sim$ é uma relação de equivalência em $A$, então o conjunto de todas as classes de equivalência definidas por $\sim$, forma uma partição de $A$.
    \end{lemma}

    \begin{proof}
        Consideraremos a seguinte notação para uma classe de equivalência: $[a] = \{b \in A \ | \ a\sim b\}$. Precisamos, então, mostrar que 
        \begin{enumerate}[label=\alph*)]
            \item cada elemento do conjunto é não vazio;
            \item os elementos são disjuntos entre si;
            \item a união de todos os elementos (classes de equivalência) formam $A$.
        \end{enumerate}
        Ora, $\forall a \in A$, $a\sim a$ é garantido pela definição de relação de equivalência. Assim, $a \in [a]$, ou seja, $[a] \not= \emptyset$.
    
        Por contraposição mostraremos que os elementos do conjunto são disjuntos entre si, i.e., se $[a] \cap [b] \not= \emptyset$, então $[a] = [b]$. Ora, se a intersecção entre $[a]$ e $[b]$ não é o conjunto vazio,
        \[\exists c \ | \ a\sim c \ \text{e} \ b\sim c,\]
        e consequentemente $c\sim a$ e $c\sim b$. Assim, pelas propriedades de simetria e transitividade,
        \[\forall x \in [a] \ | \ x\sim a \Rightarrow x\sim c \Rightarrow x\sim b,\]
        ou seja, $x \in [b]$, ou ainda, $[a] \subseteq [b]$. A mesma lógica pode ser aplicada para provar $[b] \subseteq [a]$. Portanto, $[a] = [b]$, como queríamos.
    
        Por fim, mostraremos que a união de todos os elementos formam $A$. Ora, $\forall a \in A$, todo elemento da classe de equivalência $[a]$ pertence à $A$, assim, a união de todas as classes de equivalência é subconjunto de $A$. Para o caminho inverso, tem-se que $\forall a \in A$, $a \in [a]$. Como $[a]$ pertence a união de todas as classes de equivalência, então $A$ é subconjunto da união de todas as classes de equivalência. Portanto, $A = \bigcup_{a \in A} [a]$, como queríamos, concluindo a prova.
    \end{proof}

    \begin{definition}
    \label{def:cardinalidade_classe}
        A cardinalidade do conjunto de classes laterais à esquerda de $S$ em $A$ é o \textbf{índice} de $S$ em $A$, denotado por $(A : S)$.
    \end{definition}


    \begin{proposition}
    \label{prop:cardinalidade_igual_classes}
        Todas as classes laterais de $S$ em $A$ têm a mesma cardinalidade, que é igual a $|S|$.
    \end{proposition}
    
    \begin{proof}
        Ora, $S \rightarrow aS$ é claramente uma bijeção de cada classe lateral com $S$. O mesmo pode ser afirmado sobre as classes laterais à direita.
    \end{proof}


    \begin{theorem}
    \label{theor:lagrange}
        \textbf{(Teorema de Lagrange)} Sejam $A$ um grupo finito e $S \leq A$. Então $|S|\cdot(A : S) = |A|$.
    \end{theorem}

    \begin{proof}
        Como mostrado pelo lema \ref{lemma:equivalencia_particao}, o conjunto das classes laterais à esquerda de $S$ em $A$ formam uma partição de $A$. Ademais, pela proposição \ref{prop:cardinalidade_igual_classes}, a cardinalidade de cada classe lateral é igual à cardinalidade de $S$. Assim,
        \[|A| = |S|\cdot(A : S),\]
        como queríamos mostrar.
    \end{proof}

    \begin{theorem}
    \label{theor:pequeno_teorema_fermat}
        \textbf{(Pequeno Teorema de Fermat)} Seja $p$ um número primo, então para $a$ não múltiplo de $p$
        \[a^{p-1} \equiv 1 (mod \ p).\]
    \end{theorem}
    
    \begin{proof}
        Denotando o módulo $p$ de um número $k$ por $\overline{k}$, tem-se que
        \[\overline{a} \in \mathbb{Z}/p\mathbb{Z} \backslash \{\overline{0}\}.\]
    
        Ora, $\mathbb{Z}/p\mathbb{Z}$ é um grupo com a operação de multiplicação $\odot$, tal que $\overline{1}$ é o elemento neutro e para $a,b \in \mathbb{Z}/p\mathbb{Z}$, $\overline{a}\odot\overline{b} = \overline{ab}$. A cardinalidade desse grupo é $p-1$.

        Assim, pelo teorema \ref{theor:lagrange}, $|\gen{\overline{a}}|$ divide a cardinalidade de $\mathbb{Z}/p\mathbb{Z}$. E, pelo teorema \ref{theor:ordem_finita_identidade},
        \begin{align*}
            \overline{a}^{(p-1)} &= \overline{a}^{(k\cdot|\gen{a}|)} = \overline{1}^{(k)}\\
            &= \overline{1},
        \end{align*}
        ou seja,
        \[a^{p-1} \equiv 1 (mod \ p),\]
        como queríamos.
    \end{proof}

\chapter{Subgrupos Normais e Grupos Quocientes}

Um caso importante no estudo da teoria de grupos, que nos será útil mais a frente, para um grupo $A$ e um subgrupo $H \leq A$, é quando a função com operação herdada de $A$
\begin{equation}
    (xH, yH) \mapsto xyH,
\end{equation}
para $x, y \in A$, está bem definida, isto é, quando o conjunto de subconjuntos de $A$ forma um grupo.

\begin{proposition}
\label{prop:boa_def_normalidade}
    $(xH, yH) \mapsto xyH$ estar bem definida é equivalente a $aha^{-1} \in H, \forall h \in H$, tal que $a \in A$.
\end{proposition}

\begin{proof}
    Para que a operação seja bem definida, duas entradas iguais na função devem resultar na mesma saída.
    Isto é, sejam $(x_1H,y_1H)$ e $(x_2H,y_2H)$ iguais, então $x_1y_1H = x_2y_2H$.

    Assim, sejam duas entradas iguais
    \[x_1h_1 = x_2h_2\]
    \[y_1j_1 = y_2j_2,\]
    tais que $\exists h2, \forall h1 \in H$ e $\exists j2, \forall j1 \in H$.
    \[\Rightarrow x_1 = x_2h_2h_1^{-1}\]
    \[y_1 = y_2j_2j_1^{-1}\]
    Disso, é verdade então que
    \begin{align*}(y_2^{-1}x_2^{-1})x_1y_1 &= (y_2^{-1}x_2^{-1})x_2h_2h_1^{-1}y_2j_2j_1^{-1}\\
    &= y_2^{-1}h_2h_1^{-1}y_2j_2j_1^{-1}.
    \end{align*}
    Ora, como apontado acima que a operação estar bem definida acontece quando $x_1y_1H = x_2y_2H$, i.e., $(y_2^{-1}x_2^{-1})x_1y_1 \in H$, tem-se que
    \begin{align*}(y_2^{-1}x_2^{-1})x_1y_1 \in H &\Leftrightarrow (y_2^{-1}x_2^{-1})x_2h_2h_1^{-1}y_2j_2j_1^{-1} \in H\\
    &\Leftrightarrow y_2^{-1}h_2h_1^{-1}y_2j_2j_1^{-1} \in H\\
    &\Leftrightarrow y_2^{-1}h_2h_1^{-1}y_2 \in H,
    \end{align*}
    pois $j_2j_1^{-1} \in H$.

    Tomando $h = h_2h_1^{-1}$, relembremos que $h_2$ é um elemento não arbitrário e $h_1$ é um elemento arbitrário, assim, $h$ também é um elemento arbitrário e pode-se concluir que
    \begin{equation}
    \label{eq:normal_subgroup}
        x_1y_1H = x_2y_2H \Leftrightarrow (y_2^{-1}x_2^{-1})x_1y_1 \in H \Leftrightarrow y_2^{-1}hy_2 \in H,
    \end{equation}
    $\forall h \in H$, tal que $y_2 \in A$, como queríamos mostrar.
\end{proof}

\begin{proposition}
\label{prop:normal_equivalencias}
    Seja $H \leq A$, onde $A$ é um grupo. Então as afirmações seguintes são todas equivalentes:
    \begin{enumerate}
        \setcounter{enumi}{-1}
        \item $(xH, yH) \mapsto xyH$ estar bem definida; \label{item:normal_zero}
        \item $aHa^{-1} \subseteq H$, $\forall a \in A$; \label{item:normal_first}
        \item $aHa^{-1} = H$, $\forall a \in A$; \label{item:normal_second}
        \item $aH = Ha$, $\forall a \in A$. \label{item:normal_third}
    \end{enumerate}
\end{proposition}

\begin{proof}
     Que o item~\ref{item:normal_zero} $\Leftrightarrow$ item~\ref{item:normal_first} já foi provado pela proposição \ref{prop:boa_def_normalidade}. Para mostrar que~\ref{item:normal_first} $\Rightarrow$ ~\ref{item:normal_second}, consideremos $h \in H$ e $a \in A$,
    \[h = a^{-1}(aha^{-1})a \in a^{-1}(aHa^{-1})a \subseteq a^{-1}Ha = bHb^{-1},\]
    $\forall b \in A$.

    Ademais, é imediato que~\ref{item:normal_second} $\Rightarrow$~\ref{item:normal_first}. Por fim, que~\ref{item:normal_second} $\Leftrightarrow$~\ref{item:normal_third} é óbvio uma vez que
    \[aHa^{-1} = H \Leftrightarrow aHa^{-1}a = aH = Ha.\]
\end{proof}

\begin{definition}
\label{def:subgrupo_normal}
    Chama-se de \textit{subgrupo normal} de um grupo $A$ (denotado por $H \trianglelefteq A$) um subgrupo $H \leq A$ tal que $H$ satisfaça uma (e, portanto, todas) das afirmações da proposição anterior. Nota-se ainda que como nesse caso as classes laterais à direita de $H$ e à esquerda de $H$ são iguais, elas serão chamadas simplesmente por \textit{classes laterais}.
\end{definition}

Alguns exemplos de subgrupos normais estão descritos a seguir:
\begin{enumerate}[label=Exemplo \arabic*),align=left]
    \item O subgrupo $\{e\}$ e o próprio grupo $A$ são subgrupos normais de $A$;

    \item O subgrupo (chamado de centro de $A$) \[Z(A) = \{x \in A | xa = ax, \forall a \in A\} \triangleleft A.\] Ou ainda, mais geralmente, se $H < Z(A)$, então $H \triangleleft A$.
        A prova disso vem diretamente da afirmação~\ref{item:normal_third} da proposição \ref{prop:normal_equivalencias};

    \item Se um grupo $A$ é abeliano (rever definição \ref{def:abeliano}), então todo subgrupo de $A$ é normal em $A$.
        A prova disso vem diretamente do item anterior, uma vez que o centro de um grupo abeliano é o próprio grupo.
\end{enumerate}

\begin{theorem}
\label{theor:grupo_quociente}
    Considere um subgrupo normal $H$ de um grupo $A$. Então, o conjunto das classes laterais, com operação induzida de $A$, é também um grupo. Note que esse grupo não é subgrupo de $A$.
\end{theorem}

\begin{proof}
    O conjunto das classes laterais é dado por
    \[\{aH | \ a \in A\}.\]
    Assim, sejam $a, b \in A$,
    \begin{align*}
        aH \cdot bH &= aH \cdot Hb \\
        &= aHb \\
        &= abH,
    \end{align*}
    como $ab \in A$, mostramos que a operação induzida de $A$ para o conjunto das classes laterais é fechada.
    Ademais, uma vez que a operação é induzida, temos garantida a associatividade, pois, sejam $a,b,c \in A$,
    \[(aH \cdot bH) \cdot cH = (abc)H = aH \cdot (bH \cdot cH).\]
    Agora, consideremos $e$ o elemento identidade de $A$ e $a \in A$, então
    \[eH \cdot aH = (ea)H = aH,\]
    i.e., $eH = H$ é o elemento identidade do grupo das classes laterais.
    Por fim, sejam $a \in A$ e $a^{-1} \in A$ o elemento inverso de $a$. Então,
    \[aH \cdot a^{-1}H = (aa^{-1})H = eH,\]
    i.e. $a^{-1}H$ é o elemento inverso da classe $aH$.
\end{proof}

\begin{definition}
\label{def:grupo_quociente}
    Sejam $A$ um grupo e $H \leq A$ um subgrupo, então o grupo de todas suas classes laterais (denotado por $A/H$) com a operação induzida de $A$ é chamado de \textit{grupo quociente} de $A$ por $H$.
\end{definition}

\begin{proposition}
\label{prop:grupo_quociente_e_permutadores}
    Sejam $A$ um grupo e $A'$ seu subgrupo dos comutadores, i.e., $\gen{\{xyx^{-1}y^{-1} | x,y \in A\}}$. Então,
    \begin{enumerate}
        \item $A/A'$ é abeliano; \label{item:permutadores_first}
        \item $A'$ é o menor subgrupo normal de $A$ com a propriedade do item anterior. Ou seja, se $H \triangleleft A$ é tal que $A/H$ é abeliano, então $A' \subseteq H$. \label{item:permutadores_second}
    \end{enumerate}
\end{proposition}

\begin{proof}
    Para o item~\ref{item:permutadores_first}, consideremos $a,b \in A$, então, como $(b^{-1}a^{-1}ba) \in A'$,
    \[aA' \cdot bA' = abA' = ab(b^{-1}a^{-1}ba)A' = baA' = bA' \cdot aA'. \ \square\]
    Para o item~\ref{item:permutadores_second}, suponhamos um grupo $A$ e um subgrupo $H \leq A$, tal que $A/H$ seja abeliano. Então, para $a,b \in A$,
    \[abH = aH \cdot bH = bH \cdot aH = baH.\]
    Ora, multiplicando ambas as extremidades da equação pela esquerda por $(ba)^{-1}$, tem-se
    \[a^{-1}b^{-1}abH = H,\]
    ou seja, $A' \subseteq H$.
\end{proof}

\chapter{Homomorfismos de Grupos}

\begin{definition}
\label{def:homomorfismo}
    Sejam $(A, \cdot)$ e $(\mathcal{A}, \times)$ dois grupos. A função $f: A \rightarrow \mathcal{A}$ é dita um \textit{homomorfismo} se
    \[f(a\cdot b) = f(a) \times f(b), \ \forall a,b \in A.\]
\end{definition}

Alguns exemplos de homomorfismos de grupos estão descritos a seguir:
\begin{enumerate}[label=Exemplo \arabic*),align=left]
    \item Identidade: $Id: (A,\cdot) \rightarrow (A, \cdot)$, $Id(a) = a, \ a \in A$.
    \item Trivial: $e: A \rightarrow \mathcal{A}$, $e(a) = e_\mathcal{A}, \forall a \in A$.
    \item Projeção Canônica: Sendo $H \triangleleft A$, então $\phi: A \rightarrow A/H$, $\phi(a) = aH = Ha$.
    \item Sejam $A$ é um grupo abeliano e $n \in \mathbb{Z}$ fixo, então $\phi_n: A \rightarrow A$, $\phi_n(a) = a^n$ é um homomorfismo.
    \item Seja $a \in A$ fixo, então $\mathcal{I}_a: A \rightarrow A$, $\mathcal{I}_a(x) = axa^{-1}, \ x\in A$, é um homomorfismo bijetivo.

    \begin{proof}
        Primeiramente, mostraremos que $\mathcal{I}_a$ é um homomorfismo. Ora,
        \begin{align*}
            \mathcal{I}_a(xy) &= axya^{-1} \\
            &= ax(a^{-1}a)ya^{-1} \\
            &= (axa^{-1})(aya^{-1}) \\
            &= \mathcal{I}_a(x)\mathcal{I}_a(y).
        \end{align*}

        Ademais, mostraremos que $\mathcal{I}_a$ é bijetiva. Uma função é bijetiva se, e somente se, a função admite inversa (a demonstração disso é encontrada facilmente na internet). Assim, mostraremos que $\mathcal{I}_a^{-1}(x) = a^{-1}xa$ é a inversa de $\mathcal{I}_a$. Ora, $\forall x \in A$,
        \begin{align*}
            \mathcal{I}_a^{-1}(\mathcal{I}_a(x) ) &= a^{-1}(axa^{-1})a \\
            &= (a^{-1}a)x(a^{-1}a) \\
            &= x,
        \end{align*}
        e, logo, a função é bijetora.
    \end{proof}
\end{enumerate}

Algumas propriedades importantes de homomorfismo de grupos está listada a seguir.

Seja $f:(A,\cdot) \rightarrow (\mathcal{A}, \times)$, então:
\begin{enumerate}
    \item $f(e_A) = e_\mathcal{A}$.\\
    A demonstração disso vem de que
    \[f(e_A) = f(e_A\cdot e_A) = f(e_A) \times f(e_A) \Rightarrow f(e_A) = e_\mathcal{A}.\]
    \item $f(a^{-1}) = f(a)^{-1}$.\\
    A demonstração disso vem de que $e_\mathcal{A} = f(a\cdot a^{-1}) = f(a) \times f (a^{-1})$
    \[\Rightarrow f(a)^{-1} = f(a)^{-1}\times e_\mathcal{A} = f(a)^{-1} = f(a^{-1}).\]
    \item chama-se por \textit{núcleo} do homomorfismo $f$ o subgrupo normal de $A$
    \[ker f := \{a \in A \ | \ f(a) = e_\mathcal{A}\}.\]
    A prova de que $ker f < A$ vem de que, sejam $x,y \in ker f$, então
    \[f(x\cdot y) = f(x) \times f(y) = e_\mathcal{A}.\]
    \[f(x^{-1}) = f(x)^{-1} = e_\mathcal{A}.\]
    Ademais, tem-se que $ker f \triangleleft A$ pois, para qualquer $a \in A$,
    \[f(axa^{-1}) = f(a) \times f(x) \times f(a)^{-1} = f(a) \times f(a)^{-1} = e_\mathcal{A}\]
    \[\Rightarrow axa^{-1} \in ker f. \ \square\]
    \item chama-se por \textit{imagem} de $f$ o subgrupo de $\mathcal{A}$
    \[Im(f) = \{y \in \mathcal{A} \ | \ y = f(a) \ \text{para algum $a \in A$}\}.\]
    A prova que $Im(f) < \mathcal{A}$ vem de que, sejam $x,y \in Im(f)$, então $\exists a,b \in A$ tais que
    \[x\times y = f(a) \times f(b) = f(a\cdot b) \in Im(f).\]
    \[e_\mathcal{A} = f(e_A) = f(a\cdot a^{-1}) = f(a)\times f(a^{-1}) = x \times x^{-1}\]
    \[\Rightarrow x^{-1} = f(a^{-1}) \in Im(f).\]
    \item se $H \leq A$, então $f(H) \leq \mathcal{A}$ e $f^{-1}(f(H)) = Hker f.$ \label{item:pre_imagem_H_ker_f}\\
    A prova que $f(H) \leq \mathcal{A}$ vem de que, sendo $x,y \in f(H)$, então $\exists a,b \in H$ tais que
    \[x \times y = f(a) \times f(b) = f(a\cdot b) \in f(H).\]
    \[e_\mathcal{A} = f(e_A) = f(a\cdot a^{-1}) = f(a)\times f(a^{-1}) = x \times x^{-1}\]
    \[\Rightarrow x^{-1} = f(a^{-1}) \in f(H).\]
    Ademais, provaremos que $f^{-1}(f(H)) = Hker f$.
    Sejam $h \in H$ e $k \in ker f$, então
    \[f(h\cdot k) = f(h) \times f(k) = f(h) \times e_\mathcal{A} = f(h) \in f(H)\]
    \[\Rightarrow Hker f \subseteq f^{-1}(f(H)).\]
    A inclusão contrária vem de que seja $x \in f^{-1}(f(H))$, então
    \[f(x) \in f(H),\]
    assim, $\exists h \in H$, tal que $f(x) = f(h)$. Dessa forma,
    \[f(h)^{-1}f(x) = e_\mathcal{A} \ \Rightarrow \ h^{-1}x \in ker f.\]
    Então,
    \[x = h(h^{-1}x) \in Hker f. \ \square\]
    \item $ker f = \{e_A\} \Leftrightarrow f$ é injetiva.\\
    Para a função ser injetiva, para quaisquer $a,b \in A$, se $f(a) = f(b)$, então $a = b$. Ora, sejam $a,b \in ker f$, então
    \[f(a) = f(b) = e_\mathcal{A}.\]
    Sabemos que $f(e_A) = e_\mathcal{A}$ para qualquer homomorfismo. Assim,
    \[a = b \Leftrightarrow ker f = \{e_A\}. \ \square\]
    \item se $\mathcal{O}(x)$ é finita, então $\mathcal{O}(f(x))$ divide $\mathcal{O}(x)$.\\
    A prova disso vem de que
    \[x^{\mathcal{O}(x)} = e_A.\]
    Assim,
    \[e_{\mathcal{A}} = f(e_A) = f(x^{\mathcal{O}(x)}) = f(x)^{\mathcal{O}(x)},\]
    i.e., $\mathcal{O}(f(x))$ divide $\mathcal{O}(x)$.
    \item seja $g:(\mathcal{A}, \times)\rightarrow (\mathcal{H}, \odot)$ um outro homomorfismo, então a composição
    \[g \circ f:(A,\cdot)\rightarrow (\mathcal{H},\odot)\]
    também é um homomorfismo.\\
    A prova disso vem de que
    \[g \circ f(x\cdot y) = g(f(x\cdot y)) = g(f(x) \times f(y)) =  g(f(x)) \odot g(f(y)) = (g\circ f(x)) \odot (g\circ f(y)). \ \square\]
\end{enumerate}

\begin{definition}
\label{def:isomorfismo}
    Seja $f:A \rightarrow \mathcal{A}$ um homomorfismo. $f$ é chamado de \textit{isomorfismo} se existe um homomorfismo $g:\mathcal{A} \rightarrow A$ tal que $f \circ g = id_\mathcal{A}$ e $g \circ f = id_A$. Utilizaremos a notação $A \simeq \mathcal{A}$ para denotar a relação de isomorfismo entre os grupos.
\end{definition}

\begin{proposition}
\label{prop:isomorfismo_bijetividade}
    Seja $f:(A, \cdot)\rightarrow (\mathcal{A}, \times)$ um homomorfismo, então $f$ é um isomorfismo se, e somente se, $f$ é bijetora.

    \noindent Prova:
        Pelo Teorema de Cantor-Bernstein-Schroeder\footnote{O qual diz que se existe injeção de $A\rightarrow B$ e de $B\rightarrow A$, então existe uma bijeção $A\rightarrow B$.}, tem-se que ($\Rightarrow$) é imediato. Ademais, suponhamos que $f$ é bijetiva, então $\forall x,y \in \mathcal{A}$,
        \begin{align*}
            f^{-1}(x \times y) &= f^{-1}(f(a)\times f(b))\\
            &= f^{-1}(f(a\cdot b))\\
            &= a\cdot b\\
            &= f^{-1}(x) \cdot f^{-1}(y),
        \end{align*}
        tais que $a = f^{-1}(x), b= f^{-1}(y) \in A$. Assim, mostramos que ($\Leftarrow$) também é verdadeira. $\square$
\end{proposition}

\begin{proposition}
\label{prop:ordem_f_x_igual_ordem_x}
    Seja $f:A\rightarrow\mathcal{A}$ um homomorfismo injetivo de grupos. Então
    \[\mathcal{O}(f(x)) = \mathcal{O}(x), \ \forall x \in A.\]
\end{proposition}

\begin{proof}
    A ordem de $f(x)$ é dada pela cardinalidade do subgrupo gerado por $f(x)$, i.e., $|\gen{f(x)}|$. Como já apontado ao analisar subgrupos gerados por um único elemento, podemos escrever isso também como
    \[|\{f(x)^n \ | \ n \in \mathbb{Z}\}|.\]
    Como já mostrado,
    \[\gen{x} = \{e, x, x^2, x^3, \dots, x^{\mathcal{O}(x)-1}\},\]
    onde todos os elementos são distintos.\\
    Uma vez que a função $f$ é um homomorfismo injetivo, tem-se
    \[f(x^n) = f(\underbrace{x\cdot x \cdot (\dots) \cdot x}_{\text{$n$ elementos}}) = \underbrace{f(x) \times f(x) \times (\dots) \times f(x)}_{\text{$n$ elementos}},\]
    tal que para cada entrada $a \not= b$, com $a, b \in A$, $f(a) \not= f(b)$. Assim, $\mathcal{O}(x) = \mathcal{O}(f(x))$.
\end{proof}

 \begin{theorem}
 \label{theo:primeiro_teorema_isomorfismo}
    (Primeiro Teorema do Isomorfismo). Seja $f:A \rightarrow \mathcal{A}$ um homomorfismo de grupos. Então, \[Im(f) \simeq A/ker(f).\]
 \end{theorem}
 
 \begin{proof}
     Provaremos primeiramente que o isomorfismo dado por
    \[\phi:A/ker(f) \rightarrow Im(f)\]
    \[f(x) = \phi(x \cdot ker(f))\] é bem definido, i.e., se
    \[\forall x,y \in A, \ xker(f) = yker(f) \Rightarrow \phi(xker(f)) = \phi(yker(f)).\]
    Ora,
    \begin{align*}
        &&x ker(f) &= y ker(f)\\
        &\Leftrightarrow &y^{-1}x &\in ker(f)\\
        &\Leftrightarrow &f(y^{-1}x) &= e_\mathcal{A}\\
        &\Leftrightarrow &f(y)^{-1} \times f(x) &= e_\mathcal{A}\\
        &\Leftrightarrow &f(x) &= f(y).\\
    \end{align*}
    Portanto, $\phi$ é bem definida.

    $\phi$ também é um homomorfismo uma vez que
    \[\phi(x ker(f) \cdot y ker(f)).\]
    Como $ker(f) < A$,
    \begin{align*}
        \phi(x ker(f) \cdot y ker(f)) &= \phi(xy ker(f))\\
        &= f(xy) \\
        &= f(x)f(y) \\
        &= \phi(x ker(f)) \phi(y ker(f)).
    \end{align*}
    
    Ademais, mostraremos que $\phi$ é injetiva. Ora, mostramos que
    \[f(x) = f(y) \Leftrightarrow x ker(f) = y ker(f).\]
    Assim, pela definição de $\phi$,
    \[\phi(x ker(f)) = \phi(y ker(f)) \Leftrightarrow x ker(f) = y ker(f),\]
    que é a definição de injetividade.

    Mostraremos por fim a subjetividade de $\phi$. Ora, tem-se que
    \[Im(\phi) = \phi(A ker(f)) = f(A) = Im(f),\]
    i.e., a imagem de $\phi$ é equivalente ao seu contra-domínio ($Img(f)$), como queríamos.

    O diagrama comutativo abaixo ilustra essa prova.
    
    \[\xymatrix{ & A \ar[dl]_{} \ar[dr]^{f} \ar@{}[d]|-{\circlearrowleft} \\ A/ker(f) \ar[rr]_{\phi} && Im(f) \subseteq\mathcal{A}
}\]
 \end{proof}
 
    Neste momento, enunciaremos um lema que será útil para o próximo teorema.
    
    \begin{lemma}
    \label{lemma:H_intersec_N_normal_to_H}
        Sejam $H \leq A$ um subgrupo e $N \triangleleft A$ um subgrupo normal. Então,
    \[{H \cap N} \triangleleft H.\]
    \end{lemma}
    \begin{proof}
        Uma vez que sendo ambos $H$ e $N$ subgrupos de $A$, a associatividade, o elemento identidade e inversa de cada elemento nos subgrupos são herdados de $A$. Assim, sejam $x, y \in H \cap N$, então $x, y \in H$ e $x, y \in N$. Como $xy^{-1} \in H$ e $xy^{-1} \in N$, $xy^{-1} \in H \cap N$, i.e., a operação é fechada e para todo elemento existe elemento inverso correspondente. Assim, $H \cap N$ é um subgrupo de $H$. Ademais, seja $h \in H$ e $x \in H \cap N$, então
        \[hxh^{-1} \in H,\]
        pois $x \in H$ e a operação é fechada. Além disso,
        \[hxh^{-1} \in N,\]
        pois $x \in N$ e $N$ é um subgrupo normal.
        Portanto,
        \[hxh^{-1} \in H \cap N,\]
        i.e., $H \cap N$ é subgrupo normal de $H$.
    \end{proof}

    \begin{theorem}
    \label{theo:segundo_teorema_isomorfismo}
    (Segundo Teorema do Isomorfismo). Sejam $H \leq A$ um subgrupo e $N \triangleleft A$ um subgrupo normal. Então,
    \[\frac{H}{H \cap N} \simeq \frac{HN}{N}.\]
    \end{theorem}
    
    \begin{proof}
    Temos pel o Lema \ref{lemma:H_intersec_N_normal_to_H} que
    \[H \cap N \triangleleft H.\]
    
    Agora, seja $f:H \rightarrow HN/N$, $f(h) = hN$. Mostraremos que $f$ é um homomorfismo já que
    \begin{align*}
        f(h_1h_2) &= (h_1h_2)N \\
        &= (h_1N)(h_2N) \\
        &= f(h_1)f(h_2).
    \end{align*}

    Notemos agora que
    \begin{align*}
        ker(f) &= \{h \in H \ | \ hN = e_{HN/N} = N\} \\
        &= \{h \in H \ | \ h \in N,\}
    \end{align*}
    i.e., $ker(f) = H \cap N$.

    Por fim, mostraremos que $f$ é sobrejetora. Ora, pela definição,
    \[f(h) \in HN/N \Rightarrow f(H) \subseteq HN/N.\]
    E, por sua vez, qualquer elemento de $HN/N$,
    \[hnN = hN = f(h) \in f(H) \Rightarrow HN/N \subseteq f(H)\]
    \[\Rightarrow f(H) = HN/N,\]
    i.e., $f$ é sobrejetora.

    Dessa forma, pelo Teorema \ref{theo:primeiro_teorema_isomorfismo}, tem-se que
    \[f(H) \simeq H/kerf(f) \Rightarrow \frac{HN}{N} \simeq \frac{H}{H \cap N}.\]
    \end{proof}

    \begin{theorem}
     \label{theo:terceiro_teorema_isomorfismo}
        (Terceiro Teorema do Isomorfismo). Sejam $H$ e $N$ subgrupos normais de $A$, tais que $N \subseteq H \subseteq A$. Então,
        \[\frac{A/N}{H/N} \simeq A/H.\]
     \end{theorem}

     \begin{proof}
         Primeiramente, mostraremos que $N \triangleleft H$, o que é imediato uma vez que $H \subseteq A$ e $N \triangleleft A$.

         Agora, seja a função $f: A/N \rightarrow A/H$, tal que $f(aN) = aH$. Mostraremos que ela é bem definida. Ora, sejam
         \[xN = yN.\]
         Então,
         \[y^{-1}x \in N \subseteq H,\]
         ou seja,
         \[y^{-1}x \subseteq H \Rightarrow xH = yH,\]
         e a função é bem definida.

         A prova que $f$ é um homomorfismo vem de que
         \begin{align*}
             f(xN)f(yN) &= (xH)(yH)\\
             &= xyH\\
             &= f(xyN)\\
             &= f((xN)(yN)).
         \end{align*}

         Ademais, tem-se que
         \begin{align*}
             ker(f) &= \{xN \in A/N \ | \ f(xN) = e_{A/H} \}\\
             &= \{xN \in A/N \ | \ xH = H \}\\
             &= \{xN \in A/N \ | \ x \in H \}\\
             &= H/N.
         \end{align*}

        Além disso, das propriedades do homomorfismo, sabe-se que o $ker(f) = H/N$ é subgrupo normal do domínio de $f$, isto é, $A/N$.

         Por fim, $f$ é sobrejetiva já que, sendo $aH \in A/H$, claramente,
         \[aH = f(aN) \in f(A/N).\]
         Assim,
         \[f(A/N) = A/H.\]

         Dessa forma, pelo Teorema \ref{theo:primeiro_teorema_isomorfismo}, tem-se que
         \[f(A/N) \simeq (A/N)/kerf(f) \Rightarrow A/H \simeq \frac{A/N}{H/N}.\]
     \end{proof}

    Enunciaremos agora alguns lemas sobre funções e sobre homomorfismos de grupos que nos serão úteis para o próximo teorema.

    \begin{definition}
    \label{def:pre_imagem}
        Seja uma função $f:X\rightarrow Y$, então, sendo $A \subseteq X$ e $B \subseteq Y$, definimos
        \[f(A) = \{y \in Y \ | \ y = f(a), \ \text{para algum $a \in A$}\},\]
        e definimos como sendo a \textit{pré-imagem} de $f$
        \[f^{-1}(B) = \{x \in X \ | \ f(x) \in B\}.\]
    \end{definition}
    
     \begin{lemma}
     \label{lemma:função_pre_imagem}
        Sejam uma função $f:X\rightarrow Y$ e $A \subseteq X$ e $B \subseteq Y$, é verdade que:
        \begin{enumerate}
            \item $f(f^{-1}(B)) \subseteq B$; ou ainda $f(f^{-1}(B)) = B$, sse $f$ é sobrejetora;
            \item $f^{-1}(f(A)) \supseteq A$; ou ainda $f^{-1}(f(A)) = A$, sse $f$ é injetora.
        \end{enumerate}
     \end{lemma}

     \begin{proof}
         Para a primeira afirmação, tem-se que
         \begin{align*}
             f(f^{-1}(B)) &= f(\{x \in X \ | \ f(x) \in B\})\\
             &\subseteq B.
         \end{align*}
        Ademais, se $f$ é sobrejetora, seja $b \in B$,
        \[\exists x \in X \ | \ f(x) = b.\]
        Ora, então $x \in f^{-1}(B)$ pela definição de pré-imagem, e
        \[b = f(x) \in f(f^{-1}(B)).\]
        

         Já para a segunda afirmação, tem-se
         \begin{align*}
             f^{-1}(f(A)) &= f^{-1}(\{y \in Y \ | \ y = f(a), \ \text{para algum $a \in A$}\})\\
             &= \{x \in X \ | \ f(x) \in \{y \in Y \ | \ y = f(a), \ \text{para algum $a \in A$}\}\}\\
             &= \{x \in X \ | \ f(x) = f(a), \ \text{para algum $a \in A$}\}\\
             &\supseteq A.
         \end{align*}

         Além disso, se $f$ é injetora, seja $z \in f^{-1}(f(A))$. Então, pela definição de pré-imagem,
         \[f(z) = f(a) \in f(A),\]
         para algum $a \in A$. Ora, como $f$ é injetora,
         \[z = a \in A.\]

         
     \end{proof}

     Note que se uma função $f^{-1}$ satisfaz $f(f^{-1}(B)) = B$ e $f^{-1}(f(A)) = A$, ela é também é a função inversa de $f$.

     \begin{lemma}
     \label{lemma:homomorfismo_pre_imagem}
        Seja um homomorfismo de grupos $\phi:A\rightarrow H$, então, sendo $X \leq A$ e $Y \leq H$ subgrupos, é verdade que:
        \begin{enumerate}
            \item $\phi(\phi^{-1}(Y)) = Y \cap Im(\phi)$;
            \item $\phi^{-1}(\phi(X)) = X ker(\phi)$.
        \end{enumerate}
     \end{lemma}

     \begin{proof}
         Para a primeira afirmação, tem-se que $\phi(\phi^{-1}(Y)) \subseteq Y$ pelo lema anterior (\ref{lemma:função_pre_imagem}) e ainda, por definição, $\phi(\phi^{-1}(Y)) \subseteq Im(\phi)$. Assim, \[\phi(\phi^{-1}(Y)) \subseteq Y \cap Im(\phi).\]
         Ademais, sendo $z \in Y \cap Im(\phi)$, $\exists a \in A$, tal que
         \[\phi(a) = z \in Y.\]
         Ora, então, pela definição de pré-imagem,
         \[a \in \phi^{-1}(Y)\]
         \[\Rightarrow z = \phi(a) \in \phi(\phi^{-1}(Y)).\]
         

         Já para a segunda afirmação, tem-se que $\phi^{-1}(\phi(X)) = X ker(\phi)$ pela propriedade do isomorfismo já provada (ver propriedade \ref{item:pre_imagem_H_ker_f}).
         
         E ainda, caso $\phi$ seja injetora, $ker(\phi) = \{e_A\}$ e $X ker(\phi) = X$, e assim
         \[\phi^{-1}(\phi(X)) = X.\]
         
     \end{proof}

     \begin{lemma}
    \label{lemma:projecao_canonica_sobrejetora}
        O homomorfismo (projeção canônica)
        \[\phi: A \rightarrow A/H,\]
        onde $H \triangleleft A$, é sobrejetiva.
        E, ainda,
        \[ker(\phi) = H.\]
     \end{lemma}

     \begin{proof}
         Seja $z \in A/H$, então como $z = aH$ e $a \in A$,
         \[z = aH = \phi(a) \in \phi(A),\]
         ou seja, $\phi$ é sobrejetiva.

         Além disso, provaremos que o $ker(f) = H$. Ora,
         \begin{align*}
             ker(f) &= \{a \in A \ | \ f(a) = H\}\\ 
             &= \{a \in A \ | \ aH = H\}\\ 
             &= \{a \in A \ | \ a \in H\}\\
             &= H.
         \end{align*}
     \end{proof}

     \begin{theorem}
     \label{theo:teorema_da_correspondencia}
        (Teorema da Correspondência). Seja $N \triangleleft A$ um subgrupo normal. Então, o homomorfismo
        $f:A\rightarrow A/N$ (projeção canônica) induz uma correspondência bijetiva entre o conjunto $\mathcal{L}_N$ dos subgrupos de $A$ que contêm $N$ e o conjunto $\mathcal{L}$ dos subgrupos de $A/N$, dada por:
        \[\hat{f}:V \in \mathcal{L}_N \longmapsto f(V) = V/N \in \mathcal{L}.\]
        Ademais, $\hat{f}^{-1}:\mathcal{L} \rightarrow \mathcal{L}_N$, tal que $H \mapsto f^{-1}(H)$, é função inversa de $\hat{f}$.

        Além disso, sejam $X \in \mathcal{L}_N$ e $Y \in \mathcal{L}$,
        \begin{enumerate}[label=\roman*.]
            \item $X \triangleleft A \Rightarrow f(X) \triangleleft Im(f)$;
            \item $Y \triangleleft Im(f) \Rightarrow f^{-1}(Y) \triangleleft A$.
        \end{enumerate}
    \end{theorem}

     \begin{proof}
         Mostraremos inicialmente que $V/N \leq A/N$. Sendo $x, y \in V/N$, $\exists a, b \in V$ tais que
         \[xy = (aN)(bN),\]
         e como $N \triangleleft A$ e $N \subseteq V \leq A$,
         \[xy = abN \in V/N,\]
         já que $ab \in V$.
         Ademais, 
         \[x^{-1} = (aN)^{-1} = a^{-1}N \in V/N,\]
         já que $a^{-1} \in V$.
         Assim, mostramos que
         \[V/N \leq A/N,\]
         i.e., $V/N$ é subgrupo de $A/N$ e $V/N \in \mathcal{L}$.\\
         Mostraremos agora que $\hat{f}$ é bijetora ao mostrar que a função pré-imagem \[\hat{f}^{-1}: H \mapsto f^{-1}(H) \in \mathcal{L}_N\] é função inversa de $\hat{f}$. 
         
         Ora,
         pelo lema \ref{lemma:projecao_canonica_sobrejetora}, $f$ é sobrejetora. Seja $H \in \mathcal{L}$, vamos provar agora que $\exists K \leq A$, tal que
         \[H = K/N.\]
         Mostraremos inicialmente que essa afirmação é equivalente a
         \[H \leq A/N \Rightarrow H = f^{-1}(H)/N.\]
         Seja $K := f^{-1}(H)$. Então, devemos mostrar que $K \leq A$ e $N \subseteq K$. Suponhamos $x, y \in K = f^{-1}(H)$. É verdade, portanto, que
         \[f(x), f(y) \in H.\]
         Como $H$ é um grupo,
         \begin{align*}
             & & f(x)f(y) &\in H\\
             &\Rightarrow & f(xy) &\in H\\
             &\Rightarrow & xy &\in f^{-1}(H) = K.
         \end{align*}
         Ademais, como $f(x) \in H$,
         \begin{align*}
             & & f(x)^{-1} &\in H\\
             &\Rightarrow & f(x^{-1}) &\in H\\
             &\Rightarrow & x^{-1} &\in f^{-1}(H) = K.
         \end{align*}
         E, portanto, $K \leq A$.

        Mostraremos agora que sendo $n \in N$, $n \in K$. Ora,
        \begin{align*}
            &\Rightarrow & f(n) &= nN \in H\\
            &\Rightarrow & n &\in f^{-1}(H) = K\\
            &\Rightarrow & N &\subseteq K.
        \end{align*}

        Provaremos agora que $H = f^{-1}(H)/N$.\\
        ($\supseteq$) Seja $x \in f^{-1}(H)/N$. Então $\exists y \in f^{-1}(H)$, tal que
        \[x = yN.\]
        Uma vez que $f$ é sobrejetora e $f^{-1}(H) \subseteq A$,
        \[x = yN = f(y) \in H.\]
        ($\subseteq$) Seja $h \in H \leq A/N$. Podemos escrever $h$ como
        \[h = aN = f(a),\]
        onde $a \in A$. Assim,
        \[f(a) \in H \Rightarrow a \in f^{-1}(H).\]
        Ora, então,
        \[h = aN \in f^{-1}(H)/N.\]
         
         Demonstrado que $f(K) = K/N$ para algum $K \leq A$ e $N \triangleleft K$,
         \[H = f(K) \in f(\mathcal{L}_N)\]
         \[\Rightarrow \mathcal{L} \subseteq f(\mathcal{L}_N),\]
         e $\hat{f}$ é sobrejetora. Assim, pelo lema \ref{lemma:função_pre_imagem}, 
         \[\hat{f}(\hat{f}^{-1}(\mathcal{L})) = \mathcal{L}.\]
         
         Além disso, já mostramos que $ker(f) = N$.
         Assim, sendo $V \in \mathcal{L}_N$,
         \[\hat{f}^{-1}(\hat{f}(V)) = f^{-1}(f(V)) = V ker(f) = V N = V\]
         \[\Rightarrow \hat{f}^{-1}(\hat{f}(\mathcal{L}_N)) = \mathcal{L}_N,\]
         i.e., $\hat{f}$ é injetora.
         Assim, mostramos que $\hat{f}^{-1}$ é inversa de $\hat{f}$ e $\hat{f}$ é bijetora.

         Por fim,
         
         % (i.) Sendo $X \triangleleft A$, $Xa = aX$, $\forall a \in A$, queremos mostrar que \[f(X) \triangleleft Im(f).\]
         % Ora, $\forall y \in f(X)$, como $f$ é sobrejetiva, $\exists b \in A$, tal que $y = f(b)$. Assim,
         % \[yf(X) = f(b)f(X).\]
         % Como $f$ é um homomorfismo e também pela nossa hipótese,
         % \[yf(X) = f(bX) = f(Xb) = f(X)f(b) = f(X)y,\]
         % i.e,
         % \[f(X) \triangleleft Im(f).\]

         (i.) Seja $b \in f(A) = Im(f)$. Como $f$ é sobrejetiva, $b=f(a)$, para algum $a \in A$. Então, uma vez que $X \triangleleft A$, $Xa = aX$, e
         \[bf(X) = f(a)f(X) = f(aX) = f(Xa) = f(X)f(a) = f(X)b.\]
         Portanto, pela definição de normalidade de grupos,
         \[f(X) \triangleleft Im(f).\]

         % (ii.) Sendo $Y \triangleleft Im(f)$, 
         % \[Y f(a) = f(a) Y, \ \forall a \in A.\]
         % Ademais, seja
         % \[f(af^{-1}(Y)a^{-1}).\] Então, como $f$ é um homomorfismo,
         % \[\Rightarrow f(af^{-1}(Y)a^{-1}) = f(a)f(f^{-1}(Y))f(a^{-1}).\]
         % Ainda, como $f$ é sobrejetiva,
         % \[\Rightarrow f(af^{-1}(Y)a^{-1}) = f(a)Yf(a^{-1}).\]
         % Assim, já que $Y f(a) = f(a) Y$,
         % \[f(af^{-1}(Y)a^{-1}) = Yf(a)f(a^{-1}) = Yf(a)f(a)^{-1} = Y.\]
         % \[\therefore af^{-1}(Y)a^{-1} \in f^{-1}(Y),\]
         % i.e., $f^{-1}(Y) \triangleleft A$, como queríamos.

         (ii.) Sendo $Y \triangleleft Im(f)$, queremos mostrar que $f^{-1}(Y) \triangleleft A$. Isso é equivalente a mostrar que, $\forall a \in A$,
         \[af^{-1}(Y)a^{-1} \subseteq f^{-1}(Y) \Leftrightarrow f(af^{-1}(Y)a^{-1}) \subseteq Y \Leftrightarrow f(a)Yf(a)^{-1} \subseteq Y.\]
         Ora, como $Y \triangleleft Im(f)$, 
         \[f(a)Yf(a)^{-1} \subseteq Y, \ \forall a \in A,\]
         como queríamos.
     \end{proof}


\chapter{Produto Direto de Grupos}

Veremos agora uma maneira de se obter um grupo a partir de dois grupos quaisquer.

\begin{definition}
\label{def:produto_direto}
    Sejam dois grupos $A$ e $B$, o produto direto $A \times B$ é definido em termo de \textbf{componentes} (pares ordenados $(a,b)$, tais que $a \in A$ e $b \in B$) e pelo produto cartesiano desses pares ordenados, i.e.,
    \[(a_1, b_1) \cdot (a_2, b_2) = (a_1 a_2, b_1 b_2),\]
    onde $(a_1 a_2, b_1 b_2)$ também é um elemento de $A\times B$ e, logo, a operação é fechada.
\end{definition}

\begin{proposition}
\label{proposition:direct_product_is_a_group}
    O produto direto $A\times B$ satisfaz os axiomas de grupo e, logo, é um grupo.
\end{proposition}

\begin{proof}
    Primeiramente, mostraremos que a operação é associativa. Ora,
    \begin{align*}
        ((a_1, b_1) \cdot (a_2, b_2)) \cdot (a_3, b_3) &= (a_1a_2, b_1b_2) \cdot (a_3, b_3) \\
        &= ((a_1a_2)a_3, (b_1b_2)b_3) \\
        &= (a_1(a_2a_3), b_1(b_2b_3)) \\
        &= (a_1, b_1) \cdot (a_2a_3, b_2b_3) \\
        &= (a_1, b_1) \cdot ((a_2, b_2)\cdot(a_3, b_3)).
    \end{align*}
    Mostraremos agora que a operação tem elemento inverso e identidade. Seja $a \in A$ e $b \in B$, então,
    \[(a,b) \cdot (a^{-1},b^{-1}) = (aa^{-1},bb^{-1}) = (id_A,id_B),\]
    onde $(id_A,id_B)$ é claramente a identidade da operação.
\end{proof}

Buscaremos agora descobrir as condições para que um grupo seja isomorfo a algum produto direto de grupos.

Enunciaremos para isso, um lema que nos será útil para provar essas condições.

\begin{lemma}
\label{lemma:comutabilidade_produto_direto}
    Sejam $A_1, A_2, \dots A_n$ subgrupos de um grupo $A$, então, dadas as seguintes suposições:
    \begin{enumerate}
        \item $A = A_1A_2 \dots A_n$;
        \item $A_i \triangleleft A, \forall i = 1, \dots, n$;
        \item $A_i \cap (A_1 \dots A_{i-1}A_{i+1}\dots A_n) = \{e\}, \forall i = 1, \dots, n$;
        \item $\forall a \in A, \text{existem únicos } a_1 \in A_1, a_2 \in A_2, \dots, a_n \in A_n$, tais que $a = a_1a_2\dots a_n$;
        \item $\forall i, j$, tais que $1 \leq i, j \leq n$, sendo $a_i \in A_i$ e $a_j \in A_j$, $a_ia_j = a_ja_i$;
    \end{enumerate}
    é verdade que 1., 2., 3. $\Leftrightarrow$ 4., 5..
\end{lemma}

\begin{proof}
    Começaremos mostrando que $(1., 2., 3. \Rightarrow 4.)$. Seja $a = a_1a_2\dots a_n \in A$, sendo $a_1 \in A_1, a_2 \in A_2, \dots, a_n \in A_n$. Suponhamos $b_1 \in A_1, b_2 \in A_2, \dots, b_n \in A_n$, tais que $a = b_1b_2\dots b_n$. Assim,
    \[a_1a_2\dots a_n = b_1b_2\dots b_n\]
    \begin{align*}
        \Rightarrow& & a_1 &= b_1b_2\dots b_n(a_2\dots a_n)^{-1}\\
        \Rightarrow& & a_1 &= b_1b_2\dots b_na_n^{-1}\dots a_2^{-1}\\
        \Rightarrow& & b_1^{-1}a_1 &= b_2\dots b_na_n^{-1}\dots a_2^{-1}\\
        \Rightarrow& & b_1^{-1}a_1 &\in A_2\dots A_nA_n\dots A_2.\\
    \end{align*}
    Ora, como $A_i \triangleleft A, \forall i = 1, \dots, n$, por comutatividade, i.e., $A_iA_j = A_jA_i$ para $1 \leq j \leq n$, tem-se
    \begin{align*}
        b_1^{-1}a_1 &\in A_2A_2A_3A_3\dots  A_{n-1}A_{n-1}A_nA_n\\
        &\in A_2A_3\dots  A_{n-1}A_n.\\
    \end{align*}
    Assim, pela propriedade (3.),
    \[b_1^{-1}a_1 = e \Rightarrow a_1 = b_1.\]
    Analogamente, fazemos o mesmo procedimento em tal igualdade a fim de chegar que\\ $a_2 = b_2, a_3 = b_3, \dots, a_n = b_n$. Dessa forma, mostramos indutivamente a propriedade (4.).

    Agora, mostraremos que $(1., 2., 3. \Rightarrow 5.)$. Sejam $a_i \in A_i$ e $a_j \in A_j$. Então, um elemento da forma
    \[a_ia_ja_i^{-1}a_j^{-1}\]
    é tal que
    \[(a_ia_ja_i^{-1})a_j^{-1} \in A_j,\]
    uma vez que $A_j \triangleleft A$.\\
    Ainda, como $A_i \triangleleft A$,
    \[a_i(a_ja_i^{-1}a_j^{-1}) \in A_i.\]
    Ora, então pela propriedade (3.),
    \[a_ia_ja_i^{-1}a_j^{-1} \in A_i \cap A_j = \{e\}.\]
    Assim,
    \[a_ia_j = a_ja_i,\]
    como queríamos mostrar.

        Finalmente, mostraremos que $(4., 5. \Rightarrow 1., 2., 3.)$. A propriedade (1.) é claramente satisfeita a partir da propriedade (4.)

        Para mostrar que a propriedade (2.) é satisfeita, queremos mostrar que \[aA_ia^{-1} \subseteq A_i, \forall a \in A \text{ e } \forall i = 1, \dots, n.\]
        Pela propriedade (4.) ou (1.), temos que $a$ pode ser escrito da forma
        \[a = a_1a_2\dots a_n.\]
        Então, fixado $i \in \{1, \dots, n\}$, seja $x \in A_i$. Pela comutatividade da propriedade (5.) e por $a_i x a_i^{-1} \in A_i$,
        \begin{align*}
            axa^{-1} &= a_1\dots a_i\dots a_n x (a_1\dots a_i\dots a_n)^{-1}\\
            &= a_1\dots a_i\dots a_n x a_n^{-1}\dots a_i^{-1}\dots a_1^{-1}\\
            &= a_1\dots a_i x \dots a_n a_n^{-1}\dots a_i^{-1}\dots a_1^{-1}\\
            &= a_1\dots (a_i x a_i^{-1})\dots a_1^{-1}\\
            &= a_1 a_1^{-1} a_2 a_2^{-1} \dots (a_i x a_i^{-1}) \\
            &= a_i x a_i^{-1} \\
            &\in A_i,
        \end{align*}
        como queríamos.

        Por fim, mostraremos que a propriedade (3.) é satisfeita. Para isso, seja um elemento $x \in A_i \cap A_1 \dots A_{i-1}A_{i+1}\dots A_n$. Como $x \in A_i$, então pela propriedade (4.),
        \[x = a_1a_2\dots a_n, \text{ com $a_j = e \in A_j$ para $j \not= i$ e $a_i = x$}.\]
        Ora, como $x \in A_1 \dots A_{i-1}A_{i+1}\dots A_n$,
        \[x = b_1 \dots b_{i-1} b_i b_{i+1} \dots b_n, \text{com $b_j \in A_j$ e $b_i = e$}.\]
        Utilizando ainda a propriedade (4.), existem únicos $a_1 \in A_1$, $a_2 \in A_2$, \dots, $a_n \in A_n$.
        Assim, $a_1 = b_1, a_2 = b_2, \dots, a_n = b_n$, i.e.,
        \[a_i = b_i \Rightarrow x = e,\]
        e portanto,
        \[A_i \cap (A_1 \dots A_{i-1}A_{i+1}\dots A_n) = \{e\},\]
        como queríamos mostrar.
        
    \end{proof}


    \begin{theorem}
    \label{theorem:produto_direto_isomorfo}
        Sejam $A, H_1, \dots, H_n$ grupos. O grupo $A$ é isomorfo ao grupo \\ $H_1 \times \dots \times H_n$ se, e somente se, $A$ possui os subgrupos $A_1 \simeq H_1, \dots, A_n \simeq H_n$ tais que:
        \begin{enumerate}
            \item $A = A_1A_2 \dots A_n$.
            \item $A_i \triangleleft A, \forall i = 1, \dots, n$.
            \item $A_i \cap (A_1 \dots A_{i-1}A_{i+1}\dots A_n) = \{e\}, \forall i = 1, \dots, n$.
        \end{enumerate}
    \end{theorem}

    \begin{proof}
        ($\Leftarrow$) Seja $f: A \rightarrow A_1 \times \dots \times A_n$ uma relação, tal que \[f(a) = (a_1, \dots, a_n),\]
        onde $a = a_1a_2\dots a_n$ com $a_1 \in A_1, a_2 \in A_2, \dots, a_n \in A_n$.
        
        Mostraremos agora que $f$ é uma função bem definida. Sejam $a = b \in A$, então,
        \[a = a_1a_2\dots a_n, \text{tais que $a_1 \in A_1, a_2 \in A_2, \dots, a_n \in A_n$}\]
        \[b = b_1b_2\dots b_n, \text{tais que $b_1 \in A_1, b_2 \in A_2, \dots, b_n \in A_n$}.\]
        Ora, pela propriedade (4.) do Lema \ref{lemma:comutabilidade_produto_direto},
        \[a_1 = b_1, a_2 = b_2, \dots, a_n = b_n.\]
        Assim,
        \[f(a) = f(b),\]
        como queríamos.

        Ademais, mostraremos que $f$ é um homomorfismo, aplicando a propriedade (5.) do Lema \ref{lemma:comutabilidade_produto_direto}.
        \begin{align*}
            f(ab) &= f(a_1\dots a_n b_1\dots b_n)\\
            &= f(a_1b_1\dots a_nb_n)\\
            &= (a_1b_1,\dots, a_nb_n)\\
            &= (a_1,\dots,a_n) \times (b_1,\dots,b_n)\\
            &= f(a) \times f(b).
        \end{align*}

        Finalmente, mostraremos que $f$ é uma bijeção. Ora, sejam $f(a) = f(b)$ para $a,b \in A$,
        \begin{align*}
            f(a) &= f(b)\\
            \Rightarrow(a_1,\dots,a_n) &= (b_1,\dots,b_n)
        \end{align*}
        \[\therefore a_1 = b_1, \dots, a_n = b_n,\]
        i.e., $a = b$ ($f$ é injetora).

        Para a sobrejetividade, consideremos um elemento $y \in A_1 \times \dots \times A_n$. Então, pela definição de produto direto,
        \[y = (a_1, \dots, a_n),\]
        tais que $a_1 \in A_1, a_2 \in A_2, \dots, a_n \in A_n$. Uma vez que, pela propriedade (1.), $A = A_1A_2\dots A_n,$
        \[y \in f(a),\]
        como queríamos.\\

        Finalmente, mostraremos que a função $F: A_1 \times \dots \times A_n \rightarrow H_1 \times \dots \times H_n$, tal que $F:(a_1, \dots, a_n) \mapsto (f_1(a_1), \dots, f_n(a_n))$, é uma bijeção. Seja $f_i$ a relação de isomorfismo entre $A_i$ e $H_i$, $f_i:A_i\rightarrow H_i$. A função $F$ está bem definida uma vez que, sendo $(a_1, \dots, a_n), (b_1, \dots, b_n) \in  A_1 \times \dots \times A_n$, tais que  $(a_1, \dots, a_n) = (b_1, \dots, b_n)$, é verdade que $a_i = b_i, \forall i = 1, \dots, n$. Assim, como $f_i$ é um isomorfismo,
        \[f_i(a_i) = f_i(b_i).\]
        Portanto, tem-se que
        \[F((a_1, \dots, a_n)) = (f_1(a_1), \dots, f_n(a_n)) = (f_1(b_1), \dots, f_n(b_n)) = F((b_1, \dots, b_n)).\]

        Mostraremos agora a injetividade de $F$. Sejam  $(a_1, \dots, a_n), (b_1, \dots, b_n) \in  A_1 \times \dots \times A_n$, tais que $F((a_1, \dots, a_n)) = F((b_1, \dots, b_n))$. Então, é verdade que
        \[(f_1(a_1), \dots, f_n(a_n)) = (f_1(b_1), \dots, f_n(b_n)).\]
        Temos, assim, que $f_i(a_i) = f_i(b_i)$. Ora, como $f_i$ é um isomorfismo, $a_i = b_i$ e, então,
        \[(a_1, \dots, a_n) = (b_1, \dots, b_n).\]

        Resta-nos agora mostrar a sobrejetividade de $F$. Para isso, consideremos $(h_1, \dots, h_n) \in H_1 \times \dots \times H_n$. Ora, como já assumimos que existe $f_i$ tal que $A_i \simeq H_i$, temos que
        \[\exists a_i, \ \text{tal que } f_i(a_i) = h_i, \ \text{para $i = \{1,\dots,n\}$}.\]
        Assim, podemos escrever que
        \[(h_1, \dots, h_n) = (f_1(a_1), \dots f_n(a_n)) \in F(A_1 \times \dots \times A_n).\]
        \\
        
        ($\Rightarrow$) Suponhamos $\phi: A \rightarrow H_1 \times \dots \times H_n$ um isomorfismo de grupos. Mostraremos que $A$ contém os subgrupos $A_1 \simeq H_1, \dots, A_n \simeq H_n$ (com as condições citadas no teorema). Sendo $X_i = Y_1 \times Y_2 \times \dots \times Y_n, \text{tal que $Y_j = \{e\}$ se $j \not= i$ e $Y_j = H_i$ se $j = i$}$ com $i = 1, ..., n$, definimos
        \[A_i := \phi^{-1}(X_i) \subseteq A.\]
        É verdade que $X_i$ é subgrupo de $H_1 \times \dots \times H_n$ uma vez que sendo $a, b \in X_i$, $a = (x_1, \dots, x_n)$ e $b = (y_1, \dots, y_n)$, tais que $x_j = y_j = e$ se $j \not= i$. Então,
        \begin{align*}
            ab^{-1} &= (x_1, \dots, x_n)(y_1, \dots, y_n)^{-1}\\
            &= (x_1, \dots, x_n)(y_1^{-1}, \dots, y_n^{-1})\\
            &= (x_1 y_1^{-1}, \dots, x_i y_i^{-1}, \dots, x_n y_n^{-1})\\
            &= (e, \dots, e, x_i y_i^{-1}, e, \dots, e)\\
            &\in X_i.
        \end{align*}
        Ademais, precisamos mostrar que $A_i$ é subgrupo de $A$. Dados $a, b \in A_i$, tem-se, por $\phi$ ser um isomorfismo, que $\phi(a), \phi(b) \in X_i$. Ora,
        \[\phi(a)\phi(b)^{-1} = \phi(a)\phi(b^{-1}) = \phi(ab^{-1}) \in X_i.\]
        Então,
        \[ab^{-1} \in \phi^{-1}(X_i) = A_i.\]

        Queremos mostrar ainda que $A_i \simeq H_i$. Seja, então, a função\\
        \begin{align*}
                \phi_i: X_i &\rightarrow H_i\\
                (h_1, \dots, h_n) &\mapsto h_i.
        \end{align*}
        
        $\phi_i$ é um homomorfismo já que, dados $a,b \in X_i$, $a = (x_1, \dots, x_n)$ e $b = (y_1, \dots, y_n)$, tais que $x_j = y_j = e$ se $j \not= i$,
        \begin{align*}
            \phi_i(ab) &= \phi_i((x_1, \dots, x_n)(y_1, \dots, y_n))\\
            &= \phi_i((x_1y_1, \dots, x_iy_i, \dots, x_ny_n))\\
            &= x_iy_i\\
            &= \phi_i(a)\phi_i(b).
        \end{align*}
        \par$\phi_i$ também é sobrejetiva, pois sendo $h_i \in H_i$ e
        \[x_i := (h_1, \dots, h_i, \dots, h_n) \in X_i,\]
        com $h_j = e$, se $j \not= i$, tem-se que
        \[h_i = \phi_i(x_i).\]
        Além disso, $\phi_i$ é injetiva uma vez que para $(a_1, \dots, a_n), (b_1, \dots, b_n) \in X_i$, tais que
        \[\phi_i((a_1, \dots, a_n)) = \phi_i((b_1, \dots, b_n)),\]
        é verdade que $a_i = b_i$. Então,
        \[(a_1, \dots, a_n) = (b_1, \dots, b_n),\]
        uma vez que $a_j = b_j = e$ se $j \not= i$.
        Mostrado que $\phi_i$ é um isomorfismo, então,
        \[\phi_i \circ \phi(A_i) = \phi_i(\phi(A_i)) = \phi_i(\phi(\phi^{-1}(X_i))) = \phi_i(X_i) = H_i\]
        Portanto, podemos afirmar que 
        \[A_i \simeq_{\phi_i \circ \phi} H_i,\]
        como queríamos.
        
        Quanto à propriedade (1.), seja $a \in A$, como $\phi$ é uma bijeção, $\phi^{-1}(H_1 \times \dots \times H_n) = A$. Ora, pela definição de produto direto de grupos,
        \[H_1 \times \dots \times H_n = X_1 X_2 \dots X_n.\]
        Assim,
        \begin{align*}
            \phi^{-1}(X_1 X_2 \dots X_n) &= A\\
            \phi^{-1}(X_1) \phi^{-1}(X_2) \dots \phi^{-1}(X_n) &= A\\
            A_1 A_2 \dots A_n &= A,
        \end{align*}
        como queríamos.\\
        A propriedade (3.) vem de que
        \begin{align*}
            Y &= A_i \cap (A_1\dots A_{i-1}A_{i+1}\dots A_n)\\
            &= \phi^{-1}(X_i) \cap \phi^{-1}(X_1\dots X_{i-1}X_{i+1}\dots X_n)\\
            &= \phi^{-1}(\{e\} \times \dots \times \{e\} \times H_i \times \{e\} \times \dots \times \{e\}) \cap \phi^{-1}(H_1 \times \dots \times H_{i-1} \times \{e\} \times H_{i+1} \times \dots \times H_n)\\
            &= \{e\}.
        \end{align*}
        Quanto à propriedade (2.), queremos mostrar que $aA_ia^{-1} \subseteq A_i, \forall i = 1, 2, \dots, n, \forall a \in A$.
        Ora,
        \begin{align*}
            \phi(aA_ia^{-1}) &= \phi(a)\phi(A_i)\phi(a^{-1})\\
            &= (h_1, h_2, \dots, h_n) (\{e\} \times \dots \times \{e\} \times H_i \times \{e\} \times \dots \times \{e\}) (h_1^{-1}, h_2^{-1}, \dots, h_n^{-1})\\   
            &= (h_1h_1^{-1}, \dots, h_{i-1}h_{i-1}^{-1}, h_iH_ih_i^{-1}, h_{i+1}h_{i+1}^{-1}, \dots, h_nh_n^{-1})\\
            &= \{e\} \times \dots \times \{e\} \times H_i \times \{e\} \times \dots \times \{e\}\\
            &= \phi(A_i).
        \end{align*}
        Como $\phi$ é bijetora,
        \[aA_ia^{-1} = A_i,\]
        como queríamos mostrar.
    \end{proof}

\chapter{Grupos de Permutações}

Antes de propriamente discorrer sobre grupos de permutações, enunciaremos algumas definições e proposições que serão úteis para isso.

\begin{definition}
\label{def:conjunto_subjacente}
    Chama-se de \textbf{conjunto subjacente} de um grupo $A$ o conjunto de $A$ sem a estrutura de grupo.
\end{definition}

\begin{proposition}
\label{prop:grupo_das_funcoes_bijetoras}
    Seja $C$ um conjunto. Então, $(Bij(C), \circ)$ é um grupo, onde
    \[Bij(C) = \{f:C \rightarrow C \ | \ \text{f é uma bijeção}\}\]
    e $\circ$ é a operação de composição de funções. \\
    Esse grupo é designado por $\mathcal{P}(C)$.
\end{proposition}
\begin{proof}
    A fim de provar que $\mathcal{P}(C)$ é de fato um grupo começamos mostrando que, sendo $f, g \in \mathcal{P}(C)$, então $f \circ g \in \mathcal{P}(C)$, i.e., que a operação é fechada. Como $f$ e $g$ são bijetoras pela definição do conjunto $Bij(C)$, tem-se que
    \[f \circ g: C \rightarrow C.\]
    Sendo $c,d \in C$, se $f \circ g(c) = f \circ g (d)$, então,
    \[f(g(c)) = f(g(d)) \Leftrightarrow f^{-1}(f(g(c))) = f^{-1}(f(g(d))) \Leftrightarrow g(c) = g(d) \Leftrightarrow g^{-1}(g(c)) = g^{-1}(g(d)) \Leftrightarrow c = d,\]
    ou seja, $f\circ g$ é injetiva.
    Além disso, seja $c \in C$. Ora, como $f$ é sobrejetora, $\exists x \in C$ tal que $f(x) = c$. E, como $g$ é sobrejetora, $\exists y \in C$ tal que $g(y) = x$. Assim,
    \[f(g(y)) = c,\]
    isto é, $f \circ g$ é sobrejetora. Logo, $f \circ g$ é bijetora e é elemento de $\mathcal{P}(C)$.

    Sejam $f,g,h \in \mathcal{P}(C)$, mostraremos que a operação de composição é associativa. Assim,
    \[f \circ (g \circ h) (C) = f \circ (g(h(C))) = f(g(h(C))) = (f \circ g)(h(C)) = (f \circ g) \circ h (C).\]

    Por fim, resta mostrar que o grupo admite elemento identidade e inversa. Ora, para $f \in \mathcal{P}(C)$, como $f$ é bijetora, $\exists g \in \mathcal{P}(C)$ tal que $g = f^{-1}$. Assim,
    \[f \circ g (C) = Id_{\mathcal{P}(C)},\]
    onde
    \[Id_{\mathcal{P}(C)}: c \mapsto c.\]
    
\end{proof}

Um conjunto de grupos bastante útil no estudo de grupos finitos é dos grupos de permutações. Assim, a proposição \ref{theo:teorema_de_Cayley} a seguir mostra que qualquer grupo finito é isomorfo a um subgrupo de um grupo de permutações.

\begin{theorem}
\label{theo:teorema_de_Cayley}
    (Cayley) Seja $A$ um grupo finito, tal que $n = |A|$, e $A_0$ o conjunto subjacente a $A$. Então,
    \begin{alignat*}{3}
        T:  A &\rightarrow \mathcal{P}(A_0) &\simeq S_n\\
            a &\mapsto T_a : A_0 &\isomto A_0\\
            &\phantom{a \mapsto T_a : } x &\mapsto ax,
    \end{alignat*}
    é um homomorfismo injetivo.
\end{theorem}

\begin{proof}
    % Considere $A_0 = \{x_1, x_2, \dots, x_n\}.$
    Mostraremos inicialmente que $T$ está bem definida. Sejam $a_1, a_2 \in A$ e $a_1 = a_2$, então $T_{a_1}(x_i) = a_1 x_i = a_2 x_i = T_{a_2}(x_i), \forall x_i \in A_0$. Assim, $T_{a_1} = T_{a_2}$.

    Agora, mostraremos que $T$ é um homomorfismo. Sejam $a_1, a_2 \in A$, então $T_{a_1a_2}(x_i) = (a_1a_2)x_i = a_1(a_2x_i) = T_{a_1}(T_{a_2}(x_i))$. Assim, $T_{a_1a_2} = T_{a_1}T_{a_2}$.

    Por fim, mostraremos que $T$ é injetivo. Tem-se que se $a \in ker(T)$, então $T_a = Id_{A_0}$. Ora, se $T_a = Id_{A_0}$, então $ax = x, \forall x \in A_0$. Assim, $a = e$, e portanto $ker(T) = \{e\}$, o que implica que $T$ é injetivo, pelas propriedades do homomorfismo.
\end{proof}

\begin{definition}
\label{def:r-ciclo}
    Uma permutação $\alpha \in S_n$ é um \textit{r-ciclo} se existem $a_1, a_2, \dots, a_r \in \{1, \dots, n\}$ distintos tais que $\alpha(a_1) = a_2$, $\alpha(a_2) = a_3$, $\dots$, $\alpha(a_{r-1}) = a_r$, $\alpha(a_r) = a_1$, e os demais elementos de $\{1, \dots, n\}$ são mapeados a eles mesmos. Esse \textit{r-ciclo} é denotado por $(a_1 \dots a_r)$, onde $r$ é o \textit{comprimento} do ciclo.
\end{definition}

É interessante apontar que 2-ciclos são chamados de \textit{transposições}.

Alguns \textbf{exemplos} interessantes de r-ciclos em $S_5$ são listados a seguir.
\begin{enumerate}[label=Exemplo \arabic*),align=left]
    \item $\binom{12345}{23451}$ é um 5-ciclo com uma possível representação $(12345)$.
    \item $\binom{12345}{32145}$ é uma transposição com possível representação $(13)$.
    \item O único 1-ciclo é a identidade $\binom{12345}{12345}$ com possível representação $(1)$.
\end{enumerate}

\begin{definition}
\label{def:permutacaoes_disjuntas}
    Sejam $\alpha, \beta \in S_n$ um r$_1$-ciclo e um r$_2$-ciclo. $\alpha$ e $\beta$ são ditas disjuntas se $\forall a \in \{1,2,\dots,n\}$, tem-se \textbf{ou} $\alpha(a) = a$ \textbf{ou} $\beta(a) = a$.
\end{definition}

\begin{proposition}
\label{prop:permutacoes_disjuntas_comutacao}
Dados $\alpha, \beta \in S_n$ dois ciclos disjuntos, então $\alpha \beta = \beta \alpha$.
\end{proposition}

\begin{proof}
    Ora, por definição, $\forall a \in \{1,2,\dots,n\}$, $\alpha(a) = a$ \textbf{ou} $\beta(a) = a$. Assim, para o caso em que $\alpha(a) = a$,
    \[\alpha(\beta(a)) = \beta(a) = \beta(\alpha(a)).\]
    Já para o caso em que $\beta(a) = a$,
    \[\alpha(\beta(a)) = \alpha(a) = \beta(\alpha(a)).\]
    Portanto, $\alpha \beta = \beta \alpha$.
\end{proof}

\begin{proposition}
\label{prop:ordem_r_ciclo}
    Seja $\alpha \in S_n$ um r-ciclo. Mostre que a ordem de $\alpha$ é igual a $r$.
\end{proposition}

\begin{proof}
    Para mostrarmos que a ordem de $\alpha$ é igual a $r$, mostraremos que $\alpha^r = Id = (1)$ e que $r$ é o menor inteiro positivo com essa propriedade.
    
    Primeiro, vamos mostrar que $\alpha^r = Id$. Isso significa que $\alpha^r(a_i) = a_i$ para todo $i = 1, \dots, r$. Mas isso é verdade por definição de r-ciclo, pois $\alpha(a_i) = a_{i+1}$ para $i = 1, \dots, r-1$ e $\alpha(a_r) = a_1$. Então, aplicando $\alpha$ repetidamente r vezes, temos que $\alpha^r(a_i) = a_{i+r} = a_i$, onde usamos a aritmética modular para simplificar o índice, tal que $\alpha(a_i) = a_{i+1}$, $\alpha(a_{i+1}) = a_{i+2}$, $\dots$, $\alpha(a_{i+(r-i-1)}) = a_{r}$ e $\alpha(a_r) = a_{1}$, $\dots$, $\alpha(a_{i-1}) = a_{i}$. Como existem $i$ elementos entre $a_{i-1}$ e $a_r$ e existem $r-i$ elementos entre $a_{i+(r-i-1)}$ e $a_i$, então $\alpha^r(a_i) = a_i$.

    Para mostrar que $r$ é o menor inteiro positivo que satisfaz essa propriedade, suponhamos que exista um inteiro positivo $s < r$, tal que $\alpha^s = Id$. Ora, isso contradiz a definição de r-ciclo, pois teríamos que $\alpha^s(a_i) = a_{i+s} = a_i$, o que significa que $s = 0$ ou $s = r$. Mas $s$ não pode ser zero, pois é um inteiro positivo. E $s \not= r$, pois assumimos que $s < r$. Portanto, r é o menor inteiro positivo com essa propriedade.
\end{proof}

\begin{proposition}
\label{prop:produto_ciclos_disjuntos_ordem_MMC}
    Sejam $\alpha_1, \dots, \alpha_t \in S_n$ ciclos disjuntos de comprimentos $r_1, \dots, r_t$, respectivamente, Mostre que o produto $\alpha_t \dots \alpha_1$ tem ordem igual a $MMC\{r_1, \dots, r_t\}$.
\end{proposition}

\begin{proof}
    Mostraremos primeiramente, utilizando a proposição \ref{prop:permutacoes_disjuntas_comutacao}, que, sendo $\alpha, \beta \in S_n$ dois ciclos disjuntos, então $(\alpha\beta)^k = \alpha^k \beta^k$, $\forall k \in \mathbb{N}$. Ora,
    \[(\alpha\beta)^k = \underbrace{\alpha\beta}_{\text{$k$ vezes}} = \underbrace{\alpha}_{\text{$k$ vezes}} \underbrace{\beta}_{\text{$k$ vezes}} = \alpha^k \beta^k.\]

    Ademais, com a proposição \ref{prop:ordem_r_ciclo}, tem-se que
    \[(\alpha_t \dots \alpha_1)^{MMC\{r_1, \dots, r_t\}} = \alpha_t^{MMC\{r_1, \dots, r_t\}} \dots \alpha_1^{MMC\{r_1, \dots, r_t\}} = Id,\]
    pois $MMC\{r_1, \dots, r_t\}$ é múltiplo de cada $r_i$.

    Por fim, mostraremos que $MMC\{r_1, \dots, r_t\}$ é o menor inteiro positivo que satisfaz a propriedade, e logo é a ordem do produto. Ora, seja $s < MMC\{r_1, \dots, r_t\}$ um inteiro positivo, tal que $(\alpha_t \dots \alpha_1)^s = Id$. Então, $\alpha_t^s \dots \alpha_1^s = Id$. Isso implica que $a_i^s = Id$, $\forall i = 1, \dots, t$. Todavia, isso contradiz com o fato de que a ordem de cada $\alpha_i$ ser igual a $r_i$. Logo, $MMC\{r_1, \dots, r_t\}$ é igual a ordem do produto $\alpha_t \dots \alpha_1$.
\end{proof}

\begin{proposition}
\label{prop:fatoracao_ciclos_disjuntos}
    Seja $\alpha \in S_n$ e $\alpha \not= Id$. Então, $\alpha$ é igual a um produto de ciclos disjuntos de comprimentos $\geq 2$, tal que a fatoração é única a menos da ordem dos fatores.
\end{proposition}

\begin{proof}
    Primeiramente, mostraremos a existência de um produto de ciclos disjuntos de comprimento $\geq 2$ que seja igual a $\alpha$. 
    Seja $i$ um elemento em $\{1, 2, \ldots, n\}$ tal que $\alpha(i) \neq i$.
    Considere $r \in \mathbb{N}$, tal que $r$ seja o menor valor que satisfaça $\alpha^{r}(i) = i$, i.e., criamos um ciclo $\sigma_1$.
    Agora, considere o conjunto $\{1, 2, \dots, n\} \setminus \{i, \alpha(i), \alpha^2(i), \dots, \alpha^{r-1}(i)\}$. Se este conjunto não estiver vazio, escolha um elemento $j$ nele e construa o ciclo $\sigma_2$ da mesma forma. 
    Continue este processo até que todos os elementos de $\{1, 2, \dots, n\}$ tenham sido incluídos em um ciclo. Note que cada ciclo construído terá comprimento $\geq 2$, pois $\alpha \neq \text{Id}$. Afirmamos que $\alpha = \sigma_1 \sigma_2 \dots \sigma_k$, onde $\sigma_i$ são os ciclos construídos. 
    Isso ocorre porque, para qualquer elemento $x$ em $\{1, 2, \dots, n\}$, $\alpha(x)$ é dado pela ação do ciclo que contém $x$. Como os ciclos são disjuntos, a ordem em que eles são multiplicados não importa, pela proposição \ref{prop:permutacoes_disjuntas_comutacao}.
    
    Quanto à unicidade da fatoração, suponhamos
    \[\alpha = \sigma_1 \sigma_2 \dots \sigma_k = \tau_1 \tau_2 \dots \tau_l,\]

    onde $\sigma_i$ e $\tau_j$ são ciclos disjuntos de comprimento $\geq 2$. 
    Seja $x$ um elemento em $\{1, 2, \dots, n\}$.  Sem perda de generalidade, suponha que $x$ esteja no ciclo $\sigma_1$. 
    Como $\alpha(x) = \sigma_1(x)$, então $\tau_j(x) = \sigma_1(x)$ para algum $j$. 
    Como $\tau_j$ é um ciclo, $\tau_j^s(x) = \sigma_1^s(x)$ para todo $s \in \mathbb{N}$. 
    Em particular, se $r$ é o comprimento do ciclo $\sigma_1$, então $\tau_j^r(x) = \sigma_1^r(x) = i$. 
    Portanto, o ciclo $\tau_j$ contém todos os elementos do ciclo $\sigma_1$. 
    Como $\sigma_1$ e $\tau_j$ são disjuntos, eles devem ser iguais. 
    Analogamente, podemos mostrar que cada ciclo $\sigma_i$ é igual a algum ciclo $\tau_j$. 
    
    Portanto, $k = l$ e, a menos de reordenação, $\sigma_i = \tau_i$ para todo $i$. 

    Concluímos que a fatoração de $\alpha$ em ciclos disjuntos é única a menos da ordem dos fatores.
\end{proof}

\begin{proposition}
\label{prop:tres_prop_S_n_transposicoes}
    Considere as seguintes proposições:
    \begin{enumerate}[label=\alph*)]
        \item Todo elemento de $S_n$ pode ser escrito como um produto de transposições.
        \item $S_n = \gen{(1 \ 2), (1 \ 3), \dots, (1 \ n)}$.
        \item $S_n = \gen{(1 \ 2), (2 \ 3), (3 \ 4), \dots, (n-1 \ n)}$.
    \end{enumerate}
\end{proposition}
\begin{proof}
    Para o item (a), note que o produto de transposições $(a \ b) (a \ b) = ()$ é uma maneira de escrever o elemento identidade de $S_n$, $\forall n \geq 2$.
    Agora, para $\alpha \in S_n \setminus Id$, considere a proposição \ref{prop:fatoracao_ciclos_disjuntos}. Assim,
    \[\alpha = \sigma_1 \sigma_2 \dots \sigma_k,\]
    onde $\sigma_1, \sigma_2, \dots, \sigma_k$ são ciclos disjuntos de ordem maior ou igual a $2$.
    Ora, note que, para $1 \leq i \leq k$, tem-se que, para j-ciclos sem perda de generalidade,
    \begin{align*}
        \sigma_i &= (x_{i1} x_{i2} \dots x_{ij}) \\
        &= (x_{i1} x_{ij}) (x_{i1} x_{i(j-1)}) \dots (x_{i1} x_{i2}).
    \end{align*}
    Assim, $\alpha$ pode ser reescrita como um produto de transposições,
    \[\alpha = ((x_{11} x_{1j}) (x_{11} x_{1(j-1)}) \dots (x_{11} x_{12})) \dots ((x_{k1} x_{kj}) (x_{k1} x_{k(j-1)}) \dots (x_{k1} x_{k2})),\]
    como queríamos mostrar.

    É claro que $\gen{(1 \ 2), (1 \ 3), \dots, (1 \ n)} \leq S_n$. Uma vez provado o item (a), é suficiente para o item (b) mostrar que toda transposição $(i \ j)$ pode ser escrito como um produto dos elementos do subgrupo gerado $\gen{(1 \ 2), (1 \ 3), \dots, (1 \ n)}$. Ora,
    \[(i \ j) = (1 \ i) (1 \ j) (1 \ i),\]
    se $i \not= j$, como queríamos.

    Para o item (c), é claro que $\gen{(1 \ 2), (2 \ 3), (3 \ 4), \dots, (n-1 \ n)} \leq S_n$, e é suficiente mostrar que toda transposição $(1 \ i) \in \gen{(1 \ 2), (2 \ 3), (3 \ 4), \dots, (n-1 \ n)}$. Considerando a prova do item (b), tem-se que para $i = 2$,
    \[(1 \ 2) \in \gen{(1 \ 2), (2 \ 3), (3 \ 4), \dots, (n-1 \ n)}.\]
    Por indução em $i$, $\forall i \geq 2$, e adotando-se a hipótese de que $$(1 \ i) \in \gen{(1 \ 2), (2 \ 3), (3 \ 4), \dots, (n-1 \ n)}$$,
    \[(1 \ i) (i \ i+1) (1 \ i) = (1 \ i+1).\]
    Assim, $\forall i \geq 2$,
    \[(1 \ i) \in \gen{(1 \ 2), (2 \ 3), (3 \ 4), \dots, (n-1 \ n)},\]
    como queríamos mostrar.
\end{proof}

\begin{proposition}
\label{prop:S_n_paridade_unica}
    Seja $\alpha \in S_n$, tal que $\alpha = \sigma_1 \sigma_2 \dots \sigma_k$ é uma fatoração qualquer como produto de transposições. Então a paridade de $k$ é única, isto é, ou $k$ é sempre par para qualquer fatoração de $\alpha$ em transposições, ou $k$ é sempre ímpar para qualquer fatoração de $\alpha$ em transposições.  Em outras palavras, a paridade do número de transposições numa fatoração de uma permutação $\alpha \in S_n$ é invariante. Essa paridade define a paridade (ou sinal) da permutação.
\end{proposition}
\begin{proof}
    Vide demonstração da proposição V.10.5 do livro Elementos de Álgebra \cite{GARCIA2013}.
    % Sejam $x_1, \dots, x_n$ $n$ incógnitas, e considere
    % $$ \Phi = \prod_{1 \leq i < j \leq n} (x_j - x_i). $$
    % Dada uma permutação $\sigma \in S_n$, defina uma função $f_\sigma : \{\Phi, -\Phi\} \to \{\Phi, -\Phi\}$ por
    % $$ f_\sigma(\Phi) = \prod_{1 \leq i < j \leq n} (x_{\sigma(j)} - x_{\sigma(i)}), $$
    % e $f_\sigma(-\Phi) = -f_\sigma(\Phi)$.

    % Note que, como $\sigma$ é uma permutação, $f_\sigma(\Phi) = \Phi$ ou $f_\sigma(\Phi) = -\Phi$. Além disso, se $\sigma, \rho$ são duas permutações, então $f_\sigma f_\rho = f_{\sigma \rho}$, como é fácil verificar.

    % Agora, vamos considerar o que uma transposição $\tau = (a, b)$ faz com $\Phi$. Sem perda de generalidade, seja $a < b$.

    % Os fatores $(x_j - x_i)$ com nem $i$ nem $j$ igual a $a$ ou $b$ são inalterados.

    % Para os pares com exatamente um índice em $\{a, b\}$, temos duas classes: aqueles em que o outro índice está entre $a$ e $b$, e aqueles em que o outro índice não está entre $a$ e $b$.

    % Se o outro índice está entre $a$ e $b$, então $x_j - x_a$ é enviado para $-(x_b - x_j)$ e $x_b - x_j$ é enviado para $-(x_j - x_a)$; as duas mudanças de sinal se cancelam.

    % Se o outro índice é maior que $b$, então $x_j - x_a$ e $x_j - x_b$ são trocados, sem mudanças de sinal.

    % Se o outro índice é menor que $a$, então $x_a - x_i$ e $x_b - x_i$ são trocados, sem mudanças de sinal.

    % Finalmente, o fator $x_b - x_a$ é enviado para $-(x_b - x_a)$.

    % Em resumo, se $\tau$ é uma transposição, então $f_\tau(\Phi) = -\Phi$, $f_\tau(-\Phi) = \Phi$.

    % Agora tome uma permutação arbitrária $\sigma$, e expresse-a como um produto de transposições de duas maneiras diferentes:
    % $$ \sigma = \tau_1 \cdots \tau_r = \rho_1 \cdots \rho_s. $$
    % Então
    % $$ f_\sigma(\Phi) = f_{\tau_1 \cdots \tau_r}(\Phi) = f_{\tau_1} \cdots f_{\tau_r}(\Phi) = (-1)^r \Phi $$
    % e
    % $$ f_\sigma(\Phi) = f_{\rho_1 \cdots \rho_s}(\Phi) = f_{\rho_1} \cdots f_{\rho_s}(\Phi) = (-1)^s \Phi. $$
    % Portanto, $(-1)^r \Phi = (-1)^s \Delta$, então $r$ e $s$ têm a mesma paridade: ambos ímpares, ou ambos pares.
\end{proof}

\begin{definition}
\label{def:permutacao_par}
    Seja $\alpha$ um elemento de $S_n$. $\alpha$ é dito permutação par se $\alpha$ é escrito como um produto de uma quantidade par de transposições.
\end{definition}
\begin{proposition}
\label{prop:}
    Seja $A_n = \{\alpha \in S_n \ | \ \text{$\alpha$ é permutação par}\}.$ É verdade que $A_n \leq S_n$ de índice $2$. ($A_n$ é chamado de grupo alternado ou grupo de permutações pares)
\end{proposition}

\begin{proof}
    Considere a função $\psi: S_n \rightarrow \{1, -1\}$, onde $\{1, -1\} = (\mathbb{Z}/2\mathbb{Z})^*$ (único grupo multiplicativo de dois elementos), tal que $\psi(\alpha) = 1$ se $\alpha$ é par e $\psi(\alpha) = -1$ se $\alpha$ é ímpar. Note que $\psi$ é bem definida pela proposição \ref{prop:fatoracao_ciclos_disjuntos}. Mostraremos agora que $\psi$ é um homomorfismo. Considere $\alpha, \beta \in S_n$. Para o primeiro caso e sem perda de generalidade, considere também que $\alpha$ seja par e $\beta$ seja ímpar, então
    \[\psi(\alpha \beta) =  -1 = \psi(\alpha) \psi(\beta).\]
    Considere agora o caso em que ambos $\alpha$ e $\beta$ são pares (cuja demonstração é análoga ao caso em que ambos são ímpares),
    \[\psi(\alpha \beta) = +1 = \psi(\alpha) \psi(\beta).\]
    Isso mostra que $\psi$ é um homomorfismo.
    Ora, como $S_n$ possui tantos elementos ímpares quantos pares, tem-se que $\psi$ é um homomorfismo sobrejetivo.
\end{proof}

\begin{proposition}
\label{prop:permutacoes_subgrupo_OU_indice2}
    Seja $H$ um subgrupo de $S_n$, então ou $H < A_n$ ou o índice $(H:H \cap A_n) = 2$.
\end{proposition}
\begin{proof}
    Consideremos o homomorfismo
    \[
    \psi: S_n \to \{1, -1\},
    \]
    definido por
    \[
    \psi(\alpha) = 
    \begin{cases}
    1, & \text{se } \alpha \text{ é par}, \\
    -1, & \text{se } \alpha \text{ é ímpar}.
    \end{cases}
    \]
    Restrinjimos \(\psi\) a \(H\) e consideramos a função
    \[
    \psi|_H : H \to \{1, -1\}.
    \]
    Como \(\psi\) é um homomorfismo, sua restrição \(\psi|_H\) também o é. Note que o núcleo de \(\psi|_H\) é dado por
    \[
    \ker (\psi|_H) = \{ \alpha \in H \mid \psi(\alpha)=1 \},
    \]
    isto é,
    \[
    \ker (\psi|_H)= H\cap A_n.
    \]
    
    Pelo Primeiro Teorema de Isomorfismo, temos
    \[
    H/(H\cap A_n) \simeq \operatorname{im} (\psi|_H).
    \]
    Como \(\operatorname{im} (\psi|_H)\) é um subgrupo de \(\{1,-1\}\), e este possui apenas dois subgrupos (o trivial \(\{1\}\) e o próprio \(\{1,-1\}\)), temos duas possibilidades:
    
    \begin{enumerate}
        \item Se \(\operatorname{im} (\psi|_H) = \{1\}\), então \(\psi(\alpha)=1\) para todo \(\alpha \in H\); ou seja, todos os elementos de \(H\) são pares, isto é, \(H \leq A_n\).
        \item Se \(\operatorname{im} (\psi|_H) = \{1, -1\}\), então \(H/(H\cap A_n)\) é isomorfo a \(\{1,-1\}\) e, portanto, tem exatamente 2 elementos. Assim,
        \[
        [H : H\cap A_n] = 2.
        \]
    \end{enumerate}
    
    Logo, ou \(H\) está contido em \(A_n\) ou \([H : H\cap A_n]=2\), como queríamos demonstrar.
    \end{proof}

\begin{definition}
\label{def:tipo_de_decomposicao}
    Considere $n \geq 2$. Se $\rho \in S_n$ e se $\rho = (a_{11} \dots a_{1r_1})\dots(a_{t1} \dots a_{tr_t})$ é sua decomposição em ciclos disjuntos com $r_1 \leq r_2 \leq \dots \leq r_t$, então 
    \[\{r_1,\dots,r_t\}\]
    é chamado de \textit{tipo de decomposição} de $\rho$.
\end{definition}

\begin{lemma}
\label{lemma:decomposicao_permutacao}
    Considere $n \geq 2$. Para uma permutação $\rho \in S_n$, tal que $\rho = (a_{11} \dots a_{1r_1}) \dots (a_{t1} \dots a_{tr_t})$ a sua decomposição em ciclos disjuntos, tem-se as seguintes afirmações:
    \begin{enumerate}[label=\alph*)]
        \item Se $\sigma \in S_n$, então a permutação par $\sigma\rho\sigma^{-1}$ tem a decomposição em ciclos disjuntos
            \[\sigma\rho\sigma^{-1} = (\sigma(a_{11})\dots\sigma(a_{1r_1}))\dots(\sigma(a_{t1})\dots\sigma(a_{tr_t})).\]
        \item Reciprocamente, se $\rho, \rho' \in S_n$ são permutações com o mesmo tipo de decomposição, então existe $\sigma \in S_n$ tal que $\rho' = \sigma\rho\sigma^{-1}$.
        \item Se as permutações $\rho, \rho' \in S_n$ têm o mesmo tipo de decomposição e se as permutações $\rho$ e $\rho'$ deixam pelo menos duas letras fixas, então existe $\mu \in A_n$ tal que $\rho' = \mu\rho\mu'$.
    \end{enumerate}
\end{lemma}
\begin{proof}
    \textbf{(a)} Seja $\sigma\in S_n$ e considere um dos ciclos de $\rho$, digamos,
    \[
    \gamma=(a_{11}\,a_{12}\,\dots\,a_{1r_1}).
    \]
    Seja $x=\sigma(a_{11})$. Então, temos:
    \[
    \sigma\rho\sigma^{-1}(x)=\sigma\Bigl(\rho\bigl(\sigma^{-1}(x)\bigr)\Bigr)=\sigma\Bigl(\rho(a_{11})\Bigr)=\sigma(a_{12}).
    \]
    De forma similar, para $j=1,\dots,r_1$, definindo $x_j=\sigma(a_{1j})$, obtemos
    \[
    \sigma\rho\sigma^{-1}(x_j)=\sigma\Bigl(\rho(a_{1j})\Bigr)=\sigma(a_{1,j+1}),
    \]
    com a convenção de que $a_{1,r_1+1}=a_{11}$. Assim, a ação de $\sigma\rho\sigma^{-1}$ sobre os elementos $\sigma(a_{11}),\sigma(a_{12}),\dots,\sigma(a_{1r_1})$ corresponde ao ciclo
    \begin{align*}
    (\sigma(a_{12})\,\sigma(a_{13})\,\dots\,\sigma(a_{1r_1+1})) &= (\sigma(a_{1r_1+1})\, \sigma(a_{12})\,\dots\,\sigma(a_{1r_1})) \\
    &= (\sigma(a_{11})\,\sigma(a_{12})\,\dots\,\sigma(a_{1r_1})).
    \end{align*}
    Como os ciclos da decomposição de $\rho$ são disjuntos, o mesmo argumento vale para cada um deles e, portanto,
    \[
    \sigma\rho\sigma^{-1} = \bigl(\sigma(a_{11})\,\sigma(a_{12})\,\dots\,\sigma(a_{1r_1})\bigr)
    \cdots
    \bigl(\sigma(a_{t1})\,\sigma(a_{t2})\,\dots\,\sigma(a_{tr_t})\bigr).
    \]

    \bigskip

    \textbf{(b)} Suponha que as decomposições em ciclos disjuntos de $\rho$ e $\rho'$ sejam
    \[
    \rho = (a_{11}\,a_{12}\,\dots\,a_{1r_1})\,(a_{21}\,a_{22}\,\dots\,a_{2r_2})\,\cdots\,(a_{t1}\,a_{t2}\,\dots\,a_{tr_t})
    \]
    e
    \[
    \rho' = (b_{11}\,b_{12}\,\dots\,b_{1r_1})\,(b_{21}\,b_{22}\,\dots\,b_{2r_2})\,\cdots\,(b_{t1}\,b_{t2}\,\dots\,b_{tr_t}).
    \]
    Como os ciclos correspondentes possuem o mesmo comprimento, podemos definir uma bijeção $\sigma:\{1,2,\dots,n\}\to\{1,2,\dots,n\}$ da seguinte maneira:
    \[
    \sigma(a_{ij})=b_{ij},\quad \text{para } i=1,\dots,t \text{ e } j=1,\dots,r_i,
    \]
    e, se existirem pontos fixos (isto é, elementos que não aparecem em nenhuma das notações dos ciclos), definimos $\sigma$ de modo que eles se mantenham fixos. Assim, $\sigma\in S_n$.

    Pela parte (a), temos
    \[
    \sigma\,\rho\,\sigma^{-1} = \bigl(\sigma(a_{11})\,\sigma(a_{12})\,\dots\,\sigma(a_{1r_1})\bigr)
    \cdots
    \bigl(\sigma(a_{t1})\,\sigma(a_{t2})\,\dots\,\sigma(a_{tr_t})\bigr).
    \]
    Pela definição de $\sigma$, isto é
    \[
    \sigma\,\rho\,\sigma^{-1} = (b_{11}\,b_{12}\,\dots\,b_{1r_1})\,(b_{21}\,b_{22}\,\dots\,b_{2r_2})\,\cdots\,(b_{t1}\,b_{t2}\,\dots\,b_{tr_t}) = \rho',
    \]
    o que prova o item (b).

    \bigskip

    \textbf{(c)} Pela parte (b), existe $\sigma\in S_n$ tal que
    \[
    \rho'=\sigma\,\rho\,\sigma^{-1}.
    \]
    Se $\sigma$ for par (isto é, $\sigma\in A_n$), basta tomar $\mu=\sigma$.

    Caso contrário, suponha que $\sigma$ seja ímpar. Como tanto $\rho$ quanto $\rho'$ deixam pelo menos duas letras fixas, seja $i,j\in\{1,2,\dots,n\}$ tais que
    \[
    \rho(i)=i,\quad \rho(j)=j,\quad \rho'(i)=i,\quad \rho'(j)=j.
    \]
    Considere a transposição $\tau=(i\,j)$. Pela definição de transposição, $\tau$ é ímpar e, como $i$ e $j$ são pontos fixos de $\rho$ e de $\rho'$, temos que $\tau$ comuta com ambas as permutações. Definindo
    \[
    \mu=\tau\,\sigma,
    \]
    observamos que $\mu\in A_n$, pois o produto de duas permutações ímpares é par. De fato, 
    \[
    \mu\,\rho\,\mu^{-1} = \tau\,\sigma\,\rho\,\sigma^{-1}\,\tau^{-1}.
    \]
    Como $\tau^{-1}=\tau$ e $\tau$ comuta com $\rho'$ (já que $\rho'$ fixa $i$ e $j$), obtemos
    \[
    \mu\,\rho\,\mu^{-1} = \tau\,\rho'\,\tau = \rho'.
    \]
    Portanto, existe $\mu\in A_n$ tal que $\rho'=\mu\,\rho\,\mu^{-1}$, concluindo o item (c).
\end{proof}

\begin{proposition}
\label{prop:A_n_3_ciclos}
    Para $n \geq 3$:
    \begin{enumerate}[label=\alph*)]
        \item Todo elemento de $A_n$ é um produto de 3-ciclos.
        \item Sejam $a, b \in \{1,2,\dots,n\}$, com $a \not= b$, então
        \[A_n = \gen{\{abl \ | \ l = 1,2,\dots,n; l \not= a,b \}}.\]
    \end{enumerate}
\end{proposition}
\begin{proof}
    \textbf{(a)} Seja \(\alpha\in A_n\). Pela Proposição~\ref{prop:S_n_paridade_unica}, a paridade do número de transposições numa fatoração de \(\alpha\) é invariante; portanto, por ser par, \(\alpha\) pode ser escrita como um produto de um número par de transposições:
\[
\alpha = \tau_1\tau_2\cdots\tau_{2k}, \quad \tau_i\text{ transposições}.
\]
Agrupando as transposições em pares, temos:
\[
\alpha = (\tau_1\tau_2)(\tau_3\tau_4)\cdots(\tau_{2k-1}\tau_{2k}).
\]
Mostraremos que cada produto \(\tau_{2i-1}\tau_{2i}\) pode ser escrito como produto de \(3\)-ciclos.

\textbf{Caso 1:} Se as duas transposições compartilham um elemento, isto é, se
\[
\tau_{2i-1}=(a\,b)\quad \text{e}\quad \tau_{2i}=(b\,c),
\]
com \(a,b,c\) distintos, e sem perda de generalidade, então:
\[
(a\,b)(b\,c) = (a\,b\,c),
\]
ou seja, o produto é um \(3\)-ciclo.

\textbf{Caso 2:} Se as transposições são disjuntas, isto é, se
\[
\tau_{2i-1}=(a\,b)\quad \text{e}\quad \tau_{2i}=(c\,d),
\]
com \(a,b,c,d\) todos distintos, então pode-se verificar que:
\[
(a\,b)(c\,d) = (a\,b)(a\,c)(a\,c)(c\,d) = (a\,c\,b)(a\,c\,d).
\]
Ou seja, o produto de duas transposições disjuntas pode ser escrito como produto de dois \(3\)-ciclos.

Em ambos os casos, cada par de transposições é expresso como produto de \(3\)-ciclos. Assim, \(\alpha\), que é o produto de um número par de transposições, pode ser reagrupada em produtos de \(3\)-ciclos. Concluímos que todo elemento de \(A_n\) pode ser escrito como produto de \(3\)-ciclos.

\medskip

\textbf{(b)} Seja fixos \(a,b\in\{1,2,\dots,n\}\) com \(a\neq b\). Pela parte (a), sabemos que \(A_n\) é gerado por \(3\)-ciclos. Mostraremos que, a partir dos \(3\)-ciclos da forma
\[
(a\,b\,l),\quad l\in \{1,2,\dots,n\} \text{ e } l\neq a,b,
\]
é possível obter qualquer \(3\)-ciclo de \(S_n\) (e, portanto, de \(A_n\)).

Seja \((x\,y\,z)\) um \(3\)-ciclo arbitrário. Pela Proposição~\ref{lemma:decomposicao_permutacao} (parte (b)), como todos os \(3\)-ciclos possuem o mesmo tipo de decomposição, existe \(\sigma\in S_n\) tal que
\[
(x\,y\,z)=\sigma\,(a\,b\,l)\,\sigma^{-1},
\]
para algum \(l\neq a,b\). 

Contudo, para garantir que o conjugador pertença a \(A_n\), observe que, se \(\sigma\) não for par, como \(n\ge3\) existem pelo menos três letras e, portanto, podemos compor \(\sigma\) com uma transposição que fixe \(a\) e \(b\) (por exemplo, uma transposição \(\tau=(l_1\,l_2)\) com \(l_1,l_2\notin\{a,b\}\)) de modo que \(\mu=\tau\sigma\in A_n\). Note que, como \(a\) e \(b\) estão fixos por essa transposição, temos:
\[
\mu\,(a\,b\,l)\,\mu^{-1} = \tau\,\sigma\,(a\,b\,l)\,\sigma^{-1}\,\tau^{-1} = \tau\,\bigl(\sigma\,(a\,b\,l)\,\sigma^{-1}\bigr)\,\tau^{-1} = \sigma\,(a\,b\,l)\,\sigma^{-1},
\]
pois \(\tau\) comuta com o \(3\)-ciclo \(\sigma\,(a\,b\,l)\,\sigma^{-1}\) (já que os pontos \(a\) e \(b\) permanecem fixos).

Portanto, qualquer \(3\)-ciclo é conjugado (por um conjugador par) a um \(3\)-ciclo da forma \((a\,b\,l)\). Como \(A_n\) é gerado pelos \(3\)-ciclos, conclui-se que
\[
A_n = \left\langle\,\{ (a\,b\,l) \mid l\in\{1,2,\dots,n\},\,l\neq a,b \}\,\right\rangle.
\]
\end{proof}

\begin{definition}
\label{def:grupo_simples}
    Um grupo $A$ é \textit{simples} se $A$ e $\{e\}$ são seus únicos subgrupos normais.
\end{definition}

\begin{theorem}
\label{theo:grupo_alternado_simples}
    Seja $n=3$ ou $n \geq 5$. Então $A_n$ é um grupo simples.
\end{theorem}
\begin{proof}
    Seja \(N\) um subgrupo normal não trivial de \(A_n\). Nosso objetivo é mostrar que \(N = A_n\).

    \medskip

    \textbf{Caso 1: \(n=3\).}  
    Observa-se que \(A_3\) possui exatamente 3 elementos, ou seja, \(A_3 \simeq \mathbb{Z}/3\mathbb{Z}\) é um grupo cíclico de ordem primo. Assim, os únicos subgrupos (e, em particular, os únicos subgrupos normais) de \(A_3\) são \(\{e\}\) e \(A_3\). Portanto, \(A_3\) é simples.

    \medskip

    \textbf{Caso 2: \(n\geq 5\).}  
    Seja \(1\neq \tau\in N\). Pela Proposição~\ref{prop:A_n_3_ciclos}, todo elemento de \(A_n\) pode ser escrito como um produto de \(3\)-ciclos. Existem duas possibilidades:

    \begin{enumerate}[label=\textbf{(\alph*)}]
        \item Se \(\tau\) é um \(3\)-ciclo, então temos um \(3\)-ciclo não trivial em \(N\).  
        \item Se \(\tau\) não é um \(3\)-ciclo, considere sua decomposição em ciclos disjuntos. Em algum dos fatores haverá um ciclo de comprimento diferente de 1 e, utilizando as técnicas já demonstradas (por exemplo, o fato de que o produto de duas transposições disjuntas pode ser escrito como o produto de dois \(3\)-ciclos, como na igualdade
        \[
        (a\,b)(c\,d) = (a\,c\,b)(a\,c\,d),
        \]
        que foi verificada anteriormente), pode-se encontrar, por conjugação, um \(3\)-ciclo que esteja contido em \(N\).  
    \end{enumerate}

    Em ambos os casos, concluímos que \(N\) contém um \(3\)-ciclo não trivial.

    Pela Proposição~\ref{prop:A_n_3_ciclos} (item (b)) e pelo Lema~\ref{lemma:decomposicao_permutacao} (parte (c)), todos os \(3\)-ciclos de \(A_n\) são conjugados entre si (na medida em que cada \(3\)-ciclo deixa pelo menos duas letras fixas, o que ocorre para \(n\ge5\)). Como \(N\) é normal, se contém um \(3\)-ciclo \(\rho\), então para todo \(\sigma\in A_n\) temos
    \[
    \sigma\,\rho\,\sigma^{-1} \in N.
    \]
    Ou seja, \(N\) contém todos os \(3\)-ciclos de \(A_n\).

    Por fim, como \(A_n\) é gerado pelos \(3\)-ciclos (vide Proposição~\ref{prop:A_n_3_ciclos}, item (b)), temos que \(N = A_n\).

    \medskip

    Assim, os únicos subgrupos normais de \(A_n\) são \(\{e\}\) e \(A_n\), o que, de acordo com a Definição~\ref{def:grupo_simples}, significa que \(A_n\) é simples.
\end{proof}

\begin{proposition}
    O conjunto $K \subset S_4$ dado por\
    \[K = \{id, (12)(34), (13)(24), (14)(23)\}\]
    é um grupo abeliano.
\end{proposition}

\begin{proof}
    Para provar que \(K\) é um grupo, basta verificar que \(K\) é não-vazio, fechado sob a operação de composição e que todo elemento tem seu inverso em \(K\). É claro que \(K \neq \varnothing\).

    Em seguida, mostraremos o fechamento explicitamente montando a tabela de multiplicação dos elementos de \(K\). Considere a seguinte tabela:

    \[
    \begin{array}{c|cccc}
       \cdot       & id              & (12)(34)      & (13)(24)      & (14)(23)\\ \hline
       id          & id              & (12)(34)      & (13)(24)      & (14)(23) \\
       (12)(34)    & (12)(34)       & id            & (14)(23)      & (13)(24) \\
       (13)(24)    & (13)(24)       & (14)(23)     & id            & (12)(34) \\
       (14)(23)    & (14)(23)       & (13)(24)     & (12)(34)     & id
    \end{array}
    \]

    \begin{itemize}
      \item A linha referente ao elemento \(id\) mostra que, para qualquer \(g \in K\), \(id \cdot g = g\).
      \item Observa-se que as demais linhas possuem como produto elementos que pertencem a \(K\). Por exemplo:
        \begin{itemize}
          \item \((12)(34) \cdot (13)(24) = (14)(23) \in K\).
          \item \((13)(24) \cdot (14)(23) = (12)(34) \in K\).
        \end{itemize}
    \end{itemize}

    Como \(K\) satisfaz o fechamento sob a operação e a existência de inversos, concluímos que \(K\) é um grupo. Além disso, pela tabela de multiplicação é claro pela simetria que \(K\) é abeliano, pois a multiplicação é comutativa. Assim, \(K\) é um grupo abeliano.
\end{proof}

\begin{definition}
\label{def:klein_group}
    O grupo de quatro elementos $$K = \{id, (12)(34), (13)(24), (14)(23)\}$$ é chamado de \textit{grupo de Klein}.
\end{definition}

\begin{novo}
\begin{proposition}
\label{prop:conjugacao_transpocicao}
    Sejam \(a,b,c,d\) quatro elementos distintos de \(\{1,2,3,4\}\). Então, para todo \(x\in S_4\), temos:
    \[
    x\,\bigl((a\,b)(c\,d)\bigr)\,x^{-1} = \bigl(x(a)\,x(b)\bigr)\,\bigl(x(c)\,x(d)\bigr).
    \]
    Em particular, se \(x\) é um \(3\)-ciclo, então \(x\,\bigl((a\,b)(c\,d)\bigr)\,x^{-1}\) é uma dupla transposição.
\end{proposition}
\begin{proof}
    Observe que a proposição é um caso particular do Lema~\ref{lemma:decomposicao_permutacao} (parte (a)). Assim, tem-se 
    \[x\,\bigl((a\,b)(c\,d)\bigr)\,x^{-1} = \bigl(x(a)\,x(b)\bigr)\,\bigl(x(c)\,x(d)\bigr),\]
    que é uma dupla transposição.
\end{proof}
\end{novo}

\begin{theorem}
\label{klein_subgrupos}
    Os únicos subgrupos normais do grupo alternado $A_4$ são $\{id\}$, $A_4$ e o grupo de Klein (denotado aqui por $K$). 
\end{theorem}

\begin{proof}
    Sabemos que o grupo simétrico \(S_4\) tem ordem $4! = 24$. Seus elementos podem ser classificados e expressos como produtos de transposições (2‑ciclos) da seguinte maneira:

    \begin{enumerate}
        \item \textbf{Identidade:}  
        \[
        id.
        \]
        
        \item \textbf{Transposições (2-ciclos):} 6 elementos  
        \[
        (12),\ (13),\ (14),\ (23),\ (24),\ (34).
        \]
        
        \item \textbf{3-ciclos:} 8 elementos  
        Cada 3‑ciclo pode ser escrito como o produto de 2 transposições, pois
        \[
        (a\,b\,c)=(a\,c)(a\,b).
        \]
        Por exemplo, temos:
        \[
        (123)=(13)(12),\quad (132)=(12)(13),
        \]
        e os demais 3‑ciclos:
        \[
        (124),\ (142),\ (134),\ (143),\ (234),\ (243).
        \]
        
        \item \textbf{4-ciclos:} 6 elementos  
        Cada 4‑ciclo pode ser escrito como produto de 3 transposições. Por exemplo:
        \[
        (1234)=(14)(13)(12).
        \]
        Assim, temos os 4‑ciclos:
        \[
        (1234),\ (1243),\ (1324),\ (1342),\ (1423),\ (1432).
        \]
        
        \item \textbf{Duplas transposições (produto de duas transposições disjuntas):} 3 elementos  
        Estes já estão na forma desejada:
        \[
        (12)(34),\ (13)(24),\ (14)(23).
        \]
    \end{enumerate}

    A soma dos elementos é:
    \[
    1 + 6 + 8 + 6 + 3 = 24.
    \]

    Cada elemento de \(S_4\) pode ser escrito como produto de transposições.
    Filtrando os elementos de \(S_4\) que possuem uma expressão em transposições com número par de fatores, obtemos o grupo alternado \(A_4\). Assim, temos:
    \[
    A_4 = \{\, id,\,(123),\,(132),\,(124),\,(142),\,(134),\,(143),\,(234),\,(243),\,(12)(34),\,(13)(24),\,(14)(23) \,\}.
    \]
    Para provar que \(K\triangleleft A_4\), basta mostrar que, para todo
    \(a\in A_4\) e todo \(k\in K\),
    \[
    a\,k\,a^{-1} \in K.
    \]
    Fazemos isso em três casos:

    \medskip
    \textbf{Caso 1:} \(a = e\).\\
    Temos
    \[
    e\,k\,e^{-1} = k \in K.
    \]

    \medskip
    \textbf{Caso 2:} \(a\) é uma dupla transposição, ou seja, \(a\in K\).\\
    Como \(K\) é abeliano, \(a k = k a\) e portanto
    \[
    a\,k\,a^{-1} = k\,a\,a^{-1} = k \in K.
    \]

    \medskip
    \textbf{Caso 3:} \(a\) é um 3‑ciclo.\\
    Tome
    \[
    a = (p\,q\,r),
    \]
    onde \(\{p,q,r\}\subseteq\{1,2,3,4\}\), e
    \[
    k = (i\,j)(\ell\,m)\in K,
    \quad
    \{i,j,\ell,m\}=\{1,2,3,4\}.
    \]
    Então, pela proposição \ref{prop:conjugacao_transpocicao},
    \[
    a\,k\,a^{-1}
    = \bigl(a(i)\;a(j)\bigr)\,\bigl(a(\ell)\;a(m)\bigr),
    \]
    que é novamente uma dupla de transposições disjuntas, isto é,
    \[
    a\,k\,a^{-1}\in\{(12)(34),\,(13)(24),\,(14)(23)\} = K.
    \]

    \smallskip
    Em todos os casos concluímos \(a\,k\,a^{-1}\in K\), i.e., $K$ é subgrupo normal de $A_4$.

    Resta mostrar que esses são os únicos subgrupos normais de $A_4$.

    Assim, suponhamos $H \leq A_4$, tal que $H \not= \{id\}$. Se $H$ possui 3-ciclos, digamos $(123)$, então também possui $(123)^{-1} = (132)$, assim como $(324)(132)(324)^{-1} = (124)$, pela definição de subgrupo normal.
    Ora, pela proposição \ref{prop:A_n_3_ciclos}, $A_4 = \gen{(123),(124)} = H$.

    Por fim, se $H$ não possui 3-ciclos, então deve possuir alguma dupla transposição, digamos $(12)(34)$. Assim, $H$ contêm também $(234)(12)(34)(234)^{-1} = (13)(24)$ e $(12)(34)(13)(24) = (14)(23)$. Assim, $H = K$, mostrando a unicidade dos três subgrupos.
\end{proof}

\begin{novo}
\begin{proposition}
\label{prop:S_n_subgrupos_normais}
    \begin{enumerate}[label=\alph*)]
        \item Seja $n = 3$ ou $n \geq 5$. Então os únicos subgrupos normais de $S_n$ são $\{id\}$, $A_n$ e $S_n$.
        \item Seja $n = 4$. Então os únicos subgrupos normais de $S_4$ são $\{id\}$, $A_4$, o grupo de Klein $K$ e $S_4$.
    \end{enumerate}
\end{proposition}
\begin{proposition}
\label{prop:S_n_subgrupos_normais}
    \begin{enumerate}[label=\alph*)]
        \item Seja $n = 3$ ou $n \geq 5$. Então os únicos subgrupos normais de $S_n$ são $\{id\}$, $A_n$ e $S_n$.
        \item Seja $n = 4$. Então os únicos subgrupos normais de $S_4$ são $\{id\}$, $A_4$, o grupo de Klein $K$ e $S_4$.
    \end{enumerate}
\end{proposition}
\begin{proof}
    \begin{enumerate}[label=\alph*)]
        \item É claro que $\{id\}, A_n$ e $S_n$ são subgrupos normais de $S_n$.
        
        Mostraremos agora a unicidade dos mesmos. Seja $H \triangleleft S_n$. Consideremos também o homomorfismo $\psi: H \rightarrow (\mathbb{Z}/2\mathbb{Z})^*$, já visto anteriormente com $\psi(\alpha) = 1$ se $\alpha$ é par e $\psi(\alpha) = -1$ se $\alpha$ é ímpar, para $\alpha \in S_n$.

        Note que se $H$ não contém um elemento ímpar, então $H \subseteq A_n$. Como $H$ é normal em $S_n$, então $H \triangleleft A_n$. Assim, pelo teorema \ref{theo:grupo_alternado_simples}, temos que $H = A_n$ ou $H = \{id\}$.
        Se $H$ contém um elemento ímpar, pela proposição \ref{prop:permutacoes_subgrupo_OU_indice2}, tem-se que ou o índice $(H:H\cap A_n) = 2$ ou $H < A_n$. Assim, dividiremos em casos:
        Caso $H < A_n$: Como $H \triangleleft S_n$, temos que $H$ é normal em $A_n$ e, portanto, $H = A_n$ ou $H = \{id\}$, pelo teorema \ref{theo:grupo_alternado_simples}.

        Caso $(H:H\cap A_n) = 2$: Como $H \triangleleft S_n$, então $H \cap A_n \triangleleft A_n$. E, pelo teorema \ref{theo:grupo_alternado_simples}, tem-se que $H \cap A_n = \{id\}$ ou $H \cap A_n = A_n$. Assim, $|H| = 2$ ou $H = S_n$, respectivamente. 

        Caso $|H| = 2$, ou seja, $H \cap A_n = \{id\}$. Assim, $H = \{id, \tau\}$, onde $\tau$ é uma transposição. Ora, como $H$ é um subgrupo, $\tau^2 = id \in H$. Além disso, como $H$ é normal, para qualquer $\sigma \in S_n$, $\sigma \tau \sigma^{-1} \in H$. Assim, $\sigma \tau \sigma^{-1} = id$ ou $\sigma \tau \sigma^{-1} = \tau$. No entanto, isso implica que $\tau$ comuta com todos os elementos de $S_n$, o que é uma contradição, pois $\tau$ não comuta com todas as permutações. Portanto, $H$ não pode ter ordem 2.

        Concluímos que os únicos subgrupos normais de $S_n$ são $\{id\}$, $A_n$ e $S_n$.

        \item Para $n = 4$, sabemos que $A_4$ possui exatamente três subgrupos normais: $\{id\}$, $A_4$, e o grupo de Klein $K$, conforme demonstrado no teorema \ref{klein_subgrupos}. Como $A_4 \triangleleft S_4$ e $(S_4:A_4) = 2$, pelo \hyperref[theo:teorema_da_correspondencia]{Teorema da Correspondência}, tem-se uma bijeção entre o conjunto dos subgrupos de $S_4$ que contêm $A_4$ e o conjunto dos subgrupos de $S_4/A_4$, logo, qualquer subgrupo normal de $S_4$ que não esteja contido em $A_4$ deve ser igual a $S_4$. Além disso, qualquer subgrupo normal de $S_4$ que esteja contido em $A_4$ deve ser um dos subgrupos normais de $A_4$, ou seja, $\{id\}$, $A_4$, ou $K$. Portanto, os únicos subgrupos normais de $S_4$ são $\{id\}$, $A_4$, $K$, e $S_4$.
    \end{enumerate}
\end{proof}
\end{novo}

\chapter{Grupos Solúveis}

\begin{definition}
\label{def:serie_subnormal}
    Seja \(G\) um grupo. Uma série subnormal de \(G\), denotada aqui por $(*)$ é uma sequência de subgrupos
    \[
    \{e\} = G_0 \triangleleft G_1 \triangleleft G_2 \triangleleft \cdots \triangleleft G_n = G,
    \]
    tal que cada \(G_i\) é um subgrupo normal de \(G_{i+1}\) para \(i=0,1,\dots,n-1\).
\end{definition}

Observa-se que os \textit{grupos quocientes} da série $(*)$ são os grupos $G_{i+1}/G_i$, para \(i=0,1,\dots,n-1\). 

\begin{definition}
\label{def:comprimento_serie}
    \begin{novo}
        O \textit{comprimento} de uma série subnormal $(*)$ é o número de fatores quocientes $G_{i+1}/G_i$ que não são triviais.
    \end{novo}
\end{definition}

\begin{definition}
\label{def:refinamento_serie}
    \begin{novo}
        Uma série subnormal $(*)$ é um \textit{refinamento} de outra série subnormal $(*)$ se pode ser obtida de $(*)$ pela inserção de subgrupos normais intermediários. O refinamento é dito próprio se insere ao menos um novo subgrupo normal não presente originalmente em $(*)$.
    \end{novo}
\end{definition}

\begin{definition}
\label{def:serie_decomposicao}
    \begin{novo}
        Uma série subnormal é chamada de \textit{série de composição} de $G$ se não admite refinamento próprio.
        
        Em particular, em uma série de decomposição, cada quociente é um grupo simples, pois caso contrário haveria como inserir um subgrupo normal próprio em $G_{i+1}/G_i$, refinando a série.
    \end{novo}
\end{definition}

\begin{proposition}
\label{prop:nem_todo_grupo_admite_serie_decomposicao}
    \novo{Existem grupos que não admitem série de composição.}
\end{proposition}
\begin{proof}
    \begin{novo}
        Por exemplo, o grupo aditivo $(\mathbb{Z}, +)$ não admite série de composição. Suponha, por contradição, que $\mathbb{Z}$ admita uma série de composição
        \[{0} = G_0 \triangleleft G_1 \triangleleft G_2 \triangleleft \cdots \triangleleft G_n = \mathbb{Z}.\]
        Sabemos que para toda série de composição, cada fator quociente $G_{i+1}/G_i$ é um grupo simples. Vamos analisar o primeiro subgrupo da série, $G_1$.
        Qualquer subgrupo não-trivial de $\mathbb{Z}$ é da forma $m\mathbb{Z}$ para algum inteiro $m > 0$. Assim, $G_1 = m\mathbb{Z}$ para algum $m > 0$.
        Considere agora o quociente $G_1/G_0 = G_1/{0} \cong m\mathbb{Z}$. Este quociente é isomorfo a $\mathbb{Z}$, pois $m\mathbb{Z} \cong \mathbb{Z}$ (pelo isomorfismo $f: \mathbb{Z} \to m\mathbb{Z}$ dado por $f(k) = mk$).
        Porém, $\mathbb{Z}$ não é um grupo simples, pois contém $2\mathbb{Z}$ como subgrupo normal diferente de ${0}$ e diferente do próprio $\mathbb{Z}$. Por definição, um grupo simples não pode conter subgrupos normais além do trivial e do próprio grupo.
        Isto contradiz a exigência de que todos os fatores quocientes numa série de composição sejam grupos simples.
        Portanto, $(\mathbb{Z}, +)$ não admite série de composição.
            \end{novo}
\end{proof}

\begin{proposition}
\label{prop:grupo_finito_admite_serie_decomposicao}
    \novo{Todo grupo finito $G \not= \{e\}$ admite uma série de composição.}
\end{proposition}
\begin{proof}
    \begin{novo}
        Demonstraremos por indução sobre a ordem $|G|$ do grupo.
        \textbf{Base:} Se $|G| = 1$, então $G = \{e\}$, que é o caso trivial excluído pelo enunciado.
        Se $|G| = p$ é um número primo, então $G$ é um grupo simples (pois qualquer subgrupo tem ordem 1 ou $p$ pelo Teorema de Lagrange). Neste caso, a série
        $$
        {e} = G_0 \triangleleft G_1 = G
        $$
        é uma série de composição, pois $G_1/G_0 \cong G$ é simples.\\
        \textbf{Hipótese de indução:} Suponha que todo grupo finito de ordem menor que $|G|$ admite uma série de composição.\\
        \textbf{Passo indutivo:} Para um grupo $G$ de ordem $|G| > p$, consideramos dois casos:\\
        \textit{Caso 1:} Se $G$ é simples, então a série ${e} \triangleleft G$ é uma série de composição, pois $G/{e} \cong G$ é simples.\\
        \textit{Caso 2:} Se $G$ não é simples, então existe um subgrupo normal $N$ de $G$ tal que ${e} \neq N \neq G$. Como $|N| < |G|$, pela hipótese de indução, $N$ admite uma série de composição:\\
        $$
        {e} = N_0 \triangleleft N_1 \triangleleft N_2 \triangleleft \cdots \triangleleft N_k = N
        $$
        Além disso, o grupo quociente $G/N$ também tem ordem menor que $|G|$, portanto, pela hipótese de indução, $G/N$ admite uma série de composição:
        $$
        {N}/N = H_0/N \triangleleft H_1/N \triangleleft H_2/N \triangleleft \cdots \triangleleft H_m/N = G/N
        $$
        onde $H_0 = N \triangleleft H_1 \triangleleft H_2 \triangleleft \cdots \triangleleft H_m = G$ é uma sequência de subgrupos de $G$ contendo $N$.
        Podemos então combinar estas duas séries para formar uma série subnormal para $G$:
        $$
        {e} = N_0 \triangleleft N_1 \triangleleft N_2 \triangleleft \cdots \triangleleft N_k = N = H_0 \triangleleft H_1 \triangleleft H_2 \triangleleft \cdots \triangleleft H_m = G
        $$
        Esta série é uma série de composição para $G$, pois:

        Os quocientes $N_{i+1}/N_i$ são simples, pois vêm da série de composição de $N$.
        Os quocientes $H_{j+1}/H_j$ são isomorfos a $(H_{j+1}/N)/(H_j/N)$, que são simples, pois vêm da série de composição de $G/N$.

        Portanto, todo grupo finito admite uma série de composição.
    \end{novo}
\end{proof}

\begin{definition}
\label{def:series_subnormais_equivalentes}
    \begin{novo}
        Duas séries subnormais de um grupo $G$
        $$
        {e} = G_0 \triangleleft G_1 \triangleleft G_2 \triangleleft \cdots \triangleleft G_n = G
        $$
        e
        $$
        {e} = H_0 \triangleleft H_1 \triangleleft H_2 \triangleleft \cdots \triangleleft H_m = G
        $$
        são ditas \textit{equivalentes} se $n = m$ e existe uma permutação $\pi$ do conjunto ${1, 2, \ldots, n}$ tal que $G_i/G_{i-1} \cong H_{\pi(i)}/H_{\pi(i)-1}$ para $i = 1, 2, \ldots, n$. Em outras palavras, duas séries subnormais são equivalentes se elas possuem o mesmo número de fatores quocientes e estes fatores são isomorfos entre si, possivelmente em ordem diferente.
    \end{novo}
\end{definition}
    
\bibliographystyle{plain}
\bibliography{main}

\end{document}